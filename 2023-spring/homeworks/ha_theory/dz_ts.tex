% arara: xelatex

\documentclass[12pt]{article} % размер шрифта

\usepackage{etex} % extend
\usepackage{tikz} % картинки в tikz
\usepackage{microtype} % свешивание пунктуации

\usepackage{diagbox}
\usepackage{slashbox}
\usepackage{tabularx}
\usepackage{comment}

\usepackage{tikzlings}
\usepackage{tikzducks}

\usepackage{array} % для столбцов фиксированной ширины
\usepackage{verbatim} % для вставки комментариев

\usepackage{indentfirst} % отступ в первом параграфе

\usepackage{sectsty} % для центрирования названий частей

\allsectionsfont{\centering} % приказываем центрировать все sections

\usepackage{amsmath,  amsfonts} % куча стандартных математических плюшек

\usepackage[top=1.5cm,  left=1.5cm,  right=1.5cm,  bottom=1.5cm]{geometry} % размер текста на странице

\usepackage{lastpage} % чтобы узнать номер последней страницы

\usepackage{enumitem} % дополнительные плюшки для списков
%  например \begin{enumerate}[resume] позволяет продолжить нумерацию в новом списке
\usepackage{caption} % подписи к картинкам без плавающего окружения figure

\usepackage{comment} % длинные комментарии

\usepackage{fancyhdr} % весёлые колонтитулы
\pagestyle{fancy}
\lhead{Моделирование временных рядов — 2023,  НИУ-ВШЭ}
\rhead{Домашняя работа: задачки}
\chead{}
\cfoot{}
\rfoot{}
\renewcommand{\headrulewidth}{0.4pt}
\renewcommand{\footrulewidth}{0.4pt}

\usepackage{physics}

\usepackage{url}

\usepackage{todonotes} % для вставки в документ заметок о том,  что осталось сделать
% \todo{Здесь надо коэффициенты исправить}
% \missingfigure{Здесь будет картина Последний день Помпеи}
% команда \listoftodos — печатает все поставленные \todo'шки

\usepackage{booktabs} % красивые таблицы
% заповеди из документации:
% 1. Не используйте вертикальные линии
% 2. Не используйте двойные линии
% 3. Единицы измерения помещайте в шапку таблицы
% 4. Не сокращайте .1 вместо 0.1
% 5. Повторяющееся значение повторяйте,  а не говорите "то же"

\usepackage{fontspec} % поддержка разных шрифтов
\usepackage{polyglossia} % поддержка разных языков

\setmainlanguage{russian}
\setotherlanguages{english}

\setmainfont{Linux Libertine O} % выбираем шрифт

% можно также попробовать Helvetica,  Arial,  Cambria и т.Д.

% чтобы использовать шрифт Linux Libertine на личном компе, 
% его надо предварительно скачать по ссылке
% http://www.linuxlibertine.org/index.php?id=91&L=1

\newfontfamily{\cyrillicfonttt}{Linux Libertine O}
% пояснение зачем нужно шаманство с \newfontfamily
% http://tex.stackexchange.com/questions/91507/

\AddEnumerateCounter{\asbuk}{\russian@alph}{щ} % для списков с русскими буквами
\setlist[enumerate,  2]{label=\asbuk*), ref=\asbuk*} % списки уровня 2 будут буквами а) б) \ldots 

%% эконометрические и вероятностные сокращения
\DeclareMathOperator{\Cov}{Cov}
\DeclareMathOperator{\Corr}{Corr}
\DeclareMathOperator{\Var}{Var}
\DeclareMathOperator{\E}{\mathbb{E}}
\DeclareMathOperator{\D}{Var}
\newcommand \hb{\hat{\beta}}
\newcommand \hs{\hat{\sigma}}
\newcommand \htheta{\hat{\theta}}
\newcommand \s{\sigma}
\newcommand \hy{\hat{y}}
\newcommand \hY{\hat{Y}}
\newcommand \e{\varepsilon}
\newcommand \he{\hat{\e}}
\newcommand \hVar{\widehat{\Var}}
\newcommand \hCorr{\widehat{\Corr}}
\newcommand \hCov{\widehat{\Cov}}
\newcommand \cN{\mathcal{N}}
\newcommand{\R}{\mathbb{R}}



\let\P\relax
\DeclareMathOperator{\P}{\mathbb{P}}

%\fbox{
%  \begin{minipage}{24em}
%    Фамилия,  имя и номер группы (печатными буквами):\vspace*{3ex}\par
%    \noindent\dotfill\vspace{2mm}
%  \end{minipage}
%  \begin{tabular}{@{}l p{0.8cm} p{0.8cm} p{0.8cm} p{0.8cm} p{0.8cm}@{}}
% %\toprule
% Задача & 1 & 2 & 3 & 4 & 5\\ 
% \midrule
% Балл  &  &  & & & \\
% \midrule
% %\bottomrule
% \end{tabular}
% }    


% делаем короче интервал в списках
\setlength{\itemsep}{0pt}
\setlength{\parskip}{0pt}
\setlength{\parsep}{0pt}

\begin{document}


\begin{enumerate}

    \item С помощью алгоритма DWT найдите расстояние между рядами
    \[
    x = (1, 2, 4, 2, 1) \quad \text{ и } \quad y = (1, 4, 1).    
    \]
    В качестве расстояния между отдельными наблюдениями используйте разницу $|x_i - y_j|$.
    
    \item Исследователь Василий применил дискретное преобразование Фурье (с делением на число наблюдений) к некоторому ряду,
    но записал только первые четыре значения преобразованного ряда:
    \[
    X = (3.5, -0.5 + i\sqrt{3}/2 , -0.5 + i\sqrt{3}/6, -0.5, \ldots).    
    \]
    Сначала он думал, что всё пропало, но потом понял, что данных достаточно, 
    чтобы полностью восстановить исходный ряд.
    
    \begin{enumerate}
        \item Сколько наблюдений было в исходном ряде?
        \item Восстановите исходный ряд и забытые Васей значения преобразованного ряда.
    \end{enumerate}
    
    \item Величины $X_1$, $X_2$, $X_3$ распределены независимо и равномерно на
    отрезке $[0; 1]$. Рассмотрим $L = \min\{X_1, X_2\}$, $R = \max\{X_1, X_2, X_3\}$.
    
    Найдите копулу $C$ величин $L$ и $R$. 

    \item Рассмотрим систему уравнений:
    \[
    \begin{cases}
    x_t = 1 - \frac{1}{6}x_{t-1} + \frac{2}{6}y_{t-1} + u^x_t \\
    y_t = 2 - \frac{4}{6}x_{t-1} + \frac{1}{6}y_{t-1} + u^y_t
    \end{cases},
    \]
    где $u_t$ — двумерный белый шум.
    \begin{enumerate}
    \item Есть ли у данной системы стационарное решение?
    \item Если стационарное решение есть, то найдите $\E(x_t)$ и $\E(y_t)$ для него,
    если стационарного решения нет, то для процесса с начальными условиями $x_0 = 0$, $y_0 = 0$.
    \end{enumerate}
    
    \item Василий предполагает, что $y_t$ описывается Гауссовским процессом:
    \[
    \begin{cases} 
        y_t = f(t) + u_t \\
        u_t \sim \cN(0; 4) \\
        f \sim GP(0, K) \\
        K(t_1, t_2) = 4\exp(-(t_1 - t_2)^2) \\
    \end{cases}.
    \]

    У Василия всего два наблюдения $y_0 = 0$ и $y_1 = 1$. 

    Постройте точечный и 95\%-й интервальный прогноз для $y_3$.

\end{enumerate}


\end{document}