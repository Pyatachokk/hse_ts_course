% arara: xelatex
\documentclass[12pt]{article}

\usepackage{physics}


\usepackage{tikz} % картинки в tikz
\usepackage{microtype} % свешивание пунктуации

\usepackage{array} % для столбцов фиксированной ширины

\usepackage{indentfirst} % отступ в первом параграфе

\usepackage{sectsty} % для центрирования названий частей
\allsectionsfont{\centering}

\usepackage{amsmath, amsfonts, amssymb} % куча стандартных математических плюшек

\usepackage{comment}

\usepackage[top=2cm, left=1.2cm, right=1.2cm, bottom=2cm]{geometry} % размер текста на странице

\usepackage{lastpage} % чтобы узнать номер последней страницы

\usepackage{enumitem} % дополнительные плюшки для списков
%  например \begin{enumerate}[resume] позволяет продолжить нумерацию в новом списке
\usepackage{caption}

\usepackage{url} % to use \url{link to web}

\usepackage{fancyhdr} % весёлые колонтитулы
\pagestyle{fancy}
\lhead{Анализ временных рядов}
\chead{}
\rhead{Последний праздник! 2023-06-24}
\lfoot{2023}
\cfoot{DON'T PANIC}
\rfoot{\thepage/\pageref{LastPage}}
\renewcommand{\headrulewidth}{0.4pt}
\renewcommand{\footrulewidth}{0.4pt}

\usepackage{tcolorbox} % рамочки!

\usepackage{todonotes} % для вставки в документ заметок о том, что осталось сделать
% \todo{Здесь надо коэффициенты исправить}
% \missingfigure{Здесь будет Последний день Помпеи}
% \listoftodos - печатает все поставленные \todo'шки


% более красивые таблицы
\usepackage{booktabs}
% заповеди из докупентации:
% 1. Не используйте вертикальные линни
% 2. Не используйте двойные линии
% 3. Единицы измерения - в шапку таблицы
% 4. Не сокращайте .1 вместо 0.1
% 5. Повторяющееся значение повторяйте, а не говорите "то же"



\usepackage{fontspec}
\usepackage{polyglossia}

\setmainlanguage{russian}
\setotherlanguages{english}

% download "Linux Libertine" fonts:
% http://www.linuxlibertine.org/index.php?id=91&L=1
\setmainfont{Linux Libertine O} % or Helvetica, Arial, Cambria
% why do we need \newfontfamily:
% http://tex.stackexchange.com/questions/91507/
\newfontfamily{\cyrillicfonttt}{Linux Libertine O}

\AddEnumerateCounter{\asbuk}{\russian@alph}{щ} % для списков с русскими буквами
\setlist[enumerate, 2]{label=\asbuk*),ref=\asbuk*}

%% эконометрические сокращения
\DeclareMathOperator{\Cov}{\mathbb{C}ov}
\DeclareMathOperator{\Corr}{\mathbb{C}orr}
\DeclareMathOperator{\Var}{\mathbb{V}ar}

\let\P\relax
\DeclareMathOperator{\P}{\mathbb{P}}

\DeclareMathOperator{\E}{\mathbb{E}}
% \DeclareMathOperator{\tr}{trace}
\DeclareMathOperator{\card}{card}
\DeclareMathOperator{\plim}{plim}
\DeclareMathOperator{\pCorr}{\mathrm{p}\mathbb{C}\mathrm{orr}}


\newcommand \hb{\hat{\beta}}
\newcommand \hs{\hat{\sigma}}
\newcommand \htheta{\hat{\theta}}
\newcommand \s{\sigma}
\newcommand \hy{\hat{y}}
\newcommand \hY{\hat{Y}}
\newcommand \e{\varepsilon}
\newcommand \he{\hat{\e}}
\newcommand \z{z}
\newcommand \hVar{\widehat{\Var}}
\newcommand \hCorr{\widehat{\Corr}}
\newcommand \hCov{\widehat{\Cov}}
\newcommand \cN{\mathcal{N}}
\newcommand \RR{\mathbb{R}}
\newcommand \NN{\mathbb{N}}
\newcommand{\cF}{\mathcal{F}}
\newcommand{\cH}{\mathcal{H}}


\begin{document}

% Кратко: онлайн. 

\begin{enumerate}

\item Рассмотрим процесс 
$
    y_t = \sum_{j=1}^k u_j \sin(\lambda_j t) + v_j \cos(\lambda_j t),
$
где $\lambda_j$ — некоторые константы, белые шумы $(u_j)$ и $(v_j)$ независимы и имеют одинаковую единичную дисперсию. 
    
\begin{enumerate}
        \item Найдите автоковариационную функцию $\gamma_h = \Cov(y_0, y_h)$.
        \item Стационарен ли этот процесс? 
        \item Как этот процесс классифицируется в рамках $ARMA$ модели?
\end{enumerate}


\item При построении обратного преобразования Фурье используют следующий геометрический факт:
Если $w$ — примитивный корень степени $n$ из единицы\footnote{По определению, $w^n=1$, но $w^k \neq 1$ при $k \in \{1, \ldots, n-1\}$.},
то $\sum_{j=1}^n (w^k)^j = 0$, если и только если $k$ не делится на $n$ нацело. 
Докажите этот факт геометрически или аналитически. 

Подсказка: для $k=1$ достаточно нарисовать картинку и приписать справа слово «очевидно».

\item По народным приметам определите, имеет ли стационарное решение уравнение
\[
\begin{pmatrix}
    a_t \\
    b_t \\
\end{pmatrix} =
\begin{pmatrix}
    5 \\
    6 
\end{pmatrix}  +
\begin{pmatrix}
    -2 & 2 \\
    2 & 1 \\
\end{pmatrix}
\begin{pmatrix}
    a_{t-1} \\
    b_{t-1}
\end{pmatrix} +
\begin{pmatrix}
    -1 & 2 \\
    2 & -4 \\
\end{pmatrix}
\begin{pmatrix}
    a_{t-2} \\
    b_{t-2}
\end{pmatrix} +
\begin{pmatrix}
    u_t^a \\
    u_t^b
\end{pmatrix},
\]
где $(u_t)$ — многомерный белый шум. 

\item Стационарный в сильном смысле белый шум $(u_t)$ описывается $ARCH(1)$ моделью:
$u_t = \nu_t \sigma_t$, $\nu_t \sim \cN(0;1)$, $\sigma_t^2 = 3 + 0.1 u_{t-1}^2$.
Помимо уравнений предполагают, что $\nu_t$ не зависит от $u_{t-1}$, $\nu_{t-1}$, $u_{t-2}$, $\nu_{t-2}$, \ldots. 

\begin{enumerate}
    \item Найдите безусловную дисперсию $u_t$.
    \item Докажите, что $u_t^2$ — это некий $ARMA$-процесс и выпишите его уравнение. 
\end{enumerate}


\item Перепишите $MA(1)$ модель $y_t = u_t + 0.5 u_{t-1}$, 
где $(u_t)$ — белый шум, в виде модели пространства состояний
\[
\begin{cases}
    x_t = F x_{t-1} + v_t \\
    y_t = G x_t + w_t,
\end{cases}
\]
где величины $x_0$, $(v_1, w_1)$, $(v_2, w_2)$, \ldots{ } некоррелированы. 

Нужно явно выписать, как устроен вектор $x_t$, матрицы $F$ и $G$, как связаны $v_t$ и $w_t$ с исходным шумом $u_t$.


\item В модели локального линейного тренда предполагается, что 
\[
\begin{cases}
    \ell_t = \ell_{t-1} + b_t + u_t \\
    b_t = b_{t-1} + v_t,    
\end{cases}
\]
где белые шумы $(u_t)$ и $(v_t)$ независимы.
Дополнительно предположим, что белые шумы нормальны с $\Var(u_t) = \sigma^2_u$ и  $\Var(v_t) = \sigma^2_v$.

\begin{enumerate}
    \item Найдите $\E(\ell_{t+h} \mid \ell_t, b_t, \ell_{t-1}, b_{t-1}, \ldots)$.
    \item Найдите $\Var(\ell_t)$.
\end{enumerate}

\end{enumerate}


\end{document}

