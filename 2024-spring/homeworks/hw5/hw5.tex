% arara: xelatex
\documentclass[12pt]{article}

\usepackage{physics}


\usepackage{tikz} % картинки в tikz
\usepackage{microtype} % свешивание пунктуации

\usepackage{array} % для столбцов фиксированной ширины

\usepackage{indentfirst} % отступ в первом параграфе

\usepackage{sectsty} % для центрирования названий частей
\allsectionsfont{\centering}

\usepackage{amsmath, amsfonts, amssymb} % куча стандартных математических плюшек

\usepackage{comment}

\usepackage[top=2cm, left=1.2cm, right=1.2cm, bottom=2cm]{geometry} % размер текста на странице

\usepackage{lastpage} % чтобы узнать номер последней страницы

\usepackage{enumitem} % дополнительные плюшки для списков
%  например \begin{enumerate}[resume] позволяет продолжить нумерацию в новом списке
\usepackage{caption}

\usepackage{url} % to use \url{link to web}

\usepackage{fancyhdr} % весёлые колонтитулы
\pagestyle{fancy}
\lhead{Анализ временных рядов}
\chead{}
\rhead{}
\lfoot{}
\cfoot{}
\rfoot{\thepage/\pageref{LastPage}}
\renewcommand{\headrulewidth}{0.4pt}
\renewcommand{\footrulewidth}{0.4pt}

\usepackage{tcolorbox} % рамочки!

\usepackage{todonotes} % для вставки в документ заметок о том, что осталось сделать
% \todo{Здесь надо коэффициенты исправить}
% \missingfigure{Здесь будет Последний день Помпеи}
% \listoftodos - печатает все поставленные \todo'шки


% более красивые таблицы
\usepackage{booktabs}
% заповеди из докупентации:
% 1. Не используйте вертикальные линни
% 2. Не используйте двойные линии
% 3. Единицы измерения - в шапку таблицы
% 4. Не сокращайте .1 вместо 0.1
% 5. Повторяющееся значение повторяйте, а не говорите "то же"



\usepackage{fontspec}
\usepackage{polyglossia}

\setmainlanguage{russian}
\setotherlanguages{english}

% download "Linux Libertine" fonts:
% http://www.linuxlibertine.org/index.php?id=91&L=1
\setmainfont{Linux Libertine O} % or Helvetica, Arial, Cambria
% why do we need \newfontfamily:
% http://tex.stackexchange.com/questions/91507/
\newfontfamily{\cyrillicfonttt}{Linux Libertine O}

\AddEnumerateCounter{\asbuk}{\russian@alph}{щ} % для списков с русскими буквами
\setlist[enumerate, 2]{label=\asbuk*),ref=\asbuk*}

%% эконометрические сокращения
\DeclareMathOperator{\Cov}{\mathbb{C}ov}
\DeclareMathOperator{\Corr}{\mathbb{C}orr}
\DeclareMathOperator{\Var}{\mathbb{V}ar}

\let\P\relax
\DeclareMathOperator{\P}{\mathbb{P}}

\DeclareMathOperator{\E}{\mathbb{E}}
% \DeclareMathOperator{\tr}{trace}
\DeclareMathOperator{\card}{card}
\DeclareMathOperator{\plim}{plim}
\DeclareMathOperator{\pCorr}{\mathrm{p}\mathbb{C}\mathrm{orr}}


\newcommand \hb{\hat{\beta}}
\newcommand \hs{\hat{\sigma}}
\newcommand \htheta{\hat{\theta}}
\newcommand \s{\sigma}
\newcommand \hy{\hat{y}}
\newcommand \hY{\hat{Y}}
\newcommand \e{\varepsilon}
\newcommand \he{\hat{\e}}
\newcommand \z{z}
\newcommand \hVar{\widehat{\Var}}
\newcommand \hCorr{\widehat{\Corr}}
\newcommand \hCov{\widehat{\Cov}}
\newcommand \cN{\mathcal{N}}
\newcommand \RR{\mathbb{R}}
\newcommand \NN{\mathbb{N}}
\newcommand{\cF}{\mathcal{F}}
\newcommand{\cH}{\mathcal{H}}


\begin{document}

% Кратко: очно. 

\begin{enumerate}

\item Раcсмотрим стационарную $VAR(1)$ модель
\[
\begin{pmatrix}
    a_t \\
    b_t \\
\end{pmatrix}  =
\begin{pmatrix}
    1 \\
    2 \\
\end{pmatrix} + 
\begin{pmatrix}
    0.3 & 0 \\
    0.5 & 0.6 \\
\end{pmatrix}
\begin{pmatrix}
    a_{t-1} \\
    b_{t-1}
\end{pmatrix} +
\begin{pmatrix}
    u_t \\
    v_t
\end{pmatrix},
\]
где $(u_t)$ и $(v_t)$ — некоррелированные белые шумы. 

\begin{enumerate}
    \item Является ли $(a_t)$ причиной по Гренджеру для $(b_t)$? Является ли $(b_t)$ причиной по Гренджеру для $(a_t)$?
    \item Постройте импульсную функцию реакции $(a_t)$ на единичное одноразовое изменение $(v_t)$ на 0, 1 и 2 шага. 
    \item Постройте прогноз $(a_t)$ из момента времени $t$ на два шага вперед. 
    В каком процентном соотношении дисперсия ошибки прогноза переменной $a_{t+2}$
    определяется шоками в переменных $(u_t)$ и $(v_t)$? 
    Эти проценты по-умному называются FEVD, forecast error variance decomposition. 
\end{enumerate}


\item Рассмотрим уравнение на векторный процесс $(y_t)$
\[
y_t   =
\begin{pmatrix}
    0.3 & 0 \\
    0.5 & 0.3 \\
\end{pmatrix}
y_{t-1}  +
\begin{pmatrix}
    0.04 & 0 \\
    0 & 0.04 \\
\end{pmatrix}
y_{t-2} +
\begin{pmatrix}
    u_t \\
    v_t
\end{pmatrix},
\]
где $(u_t)$ и $(v_t)$ — некоррелированные белые шумы. 
\begin{enumerate}
    \item Имеет ли это уравнение стационарное решение? Правда ли, что в этом стационарном решении $y_t$ не зависит от будущих значений $u_{t+s}$ и $v_{t+s}$?
    \item Рассмотрим векторный процесс
    $
        w_t = \begin{pmatrix}
            y_t \\
            y_{t-1}
        \end{pmatrix}.
    $
    Докажите, что $(w_t)$ удовлятворяет некоторому $VAR(1)$ уравнению вида 
    $
    w_t = A w_{t-1} + \nu_t.    
    $
    Явно укажите вид матрицы $A$ и вектора ошибок $\nu_t$.
\end{enumerate}

\item Рассмотрим копулу Клейтона, 
\[
C(u_1, u_2) = \max\{0, (u_1^t + u_2^t - 1)^{1/t} \},    
\]
где $t \leq 1$. 
\begin{enumerate}
    \item Как связаны между собой величины $U_1$ и $U_t$ при $t\to 0$?
    \item Найдите условное распределение $U_2$ при $U_1 = 0.5$ и $t = -1$.
\end{enumerate}

\item Рассмотрим стационарный процесс $AR(1)-GARCH(1,1)$:
\[
\begin{cases}
y_t = 2 + 0.3 y_{t-1} + u_t \\
u_t = \sigma_t \cdot \nu_t \\
\nu_t \sim \cN(0;1) \\
\sigma^2_t = 2 + 0.4 \sigma^2_{t-1} + 0.2u_{t-1}^2
\end{cases}.
\]

Найдите $\Var(u_t \mid \cF_{t-1})$, $\Var(y_t \mid \cF_{t-1})$, $\Var(u_t)$, $\Var(y_t)$.

\item Рассмотрим стационарный процесс со стохастической волатильностью $ARSV(1)$:
\[
\begin{cases}
u_t = \exp(h_t/2) \nu_t \\
\nu_t \sim \cN(0;1) \\
h_t = 2 + 0.4 h_{t-1} + \eta_t \\
\eta_t \sim \cN(0;\sigma^2) \\
\end{cases},
\]
где последовательности ошибок $(\nu_t)$ и $(\eta_t)$ независимы. 

\begin{enumerate}
    \item Найдите $\Var(u_t \mid \cF_{t-1})$, $\Var(u_t)$.
    \item Может оказаться, что в пункте (а) условная дисперсия будет меньше безусловной?
\end{enumerate}


\end{enumerate}


\end{document}

