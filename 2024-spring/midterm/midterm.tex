% arara: xelatex
\documentclass[12pt]{article}

\usepackage{physics}


\usepackage{tikz} % картинки в tikz
\usepackage{microtype} % свешивание пунктуации

\usepackage{array} % для столбцов фиксированной ширины

\usepackage{indentfirst} % отступ в первом параграфе

\usepackage{sectsty} % для центрирования названий частей
\allsectionsfont{\centering}

\usepackage{amsmath, amsfonts, amssymb} % куча стандартных математических плюшек

\usepackage{comment}

\usepackage[top=2cm, left=1.2cm, right=1.2cm, bottom=2cm]{geometry} % размер текста на странице

\usepackage{lastpage} % чтобы узнать номер последней страницы

\usepackage{enumitem} % дополнительные плюшки для списков
%  например \begin{enumerate}[resume] позволяет продолжить нумерацию в новом списке
\usepackage{caption}

\usepackage{url} % to use \url{link to web}

\usepackage{fancyhdr} % весёлые колонтитулы
\pagestyle{fancy}
\lhead{Анализ временных рядов}
\chead{}
\rhead{2024-04-06}
\lfoot{}
\cfoot{DON'T PANIC}
\rfoot{\thepage/\pageref{LastPage}}
\renewcommand{\headrulewidth}{0.4pt}
\renewcommand{\footrulewidth}{0.4pt}

\usepackage{tcolorbox} % рамочки!

\usepackage{todonotes} % для вставки в документ заметок о том, что осталось сделать
% \todo{Здесь надо коэффициенты исправить}
% \missingfigure{Здесь будет Последний день Помпеи}
% \listoftodos - печатает все поставленные \todo'шки


% более красивые таблицы
\usepackage{booktabs}
% заповеди из докупентации:
% 1. Не используйте вертикальные линни
% 2. Не используйте двойные линии
% 3. Единицы измерения - в шапку таблицы
% 4. Не сокращайте .1 вместо 0.1
% 5. Повторяющееся значение повторяйте, а не говорите "то же"



\usepackage{fontspec}
\usepackage{polyglossia}

\setmainlanguage{russian}
\setotherlanguages{english}

% download "Linux Libertine" fonts:
% http://www.linuxlibertine.org/index.php?id=91&L=1
\setmainfont{Linux Libertine O} % or Helvetica, Arial, Cambria
% why do we need \newfontfamily:
% http://tex.stackexchange.com/questions/91507/
\newfontfamily{\cyrillicfonttt}{Linux Libertine O}

\AddEnumerateCounter{\asbuk}{\russian@alph}{щ} % для списков с русскими буквами
\setlist[enumerate, 2]{label=\asbuk*),ref=\asbuk*}

%% эконометрические сокращения
\DeclareMathOperator{\Cov}{\mathbb{C}ov}
\DeclareMathOperator{\Corr}{\mathbb{C}orr}
\DeclareMathOperator{\Var}{\mathbb{V}ar}

\let\P\relax
\DeclareMathOperator{\P}{\mathbb{P}}

\DeclareMathOperator{\E}{\mathbb{E}}
% \DeclareMathOperator{\tr}{trace}
\DeclareMathOperator{\card}{card}
\DeclareMathOperator{\plim}{plim}
\DeclareMathOperator{\pCorr}{\mathrm{p}\mathbb{C}\mathrm{orr}}


\newcommand \hb{\hat{\beta}}
\newcommand \hs{\hat{\sigma}}
\newcommand \htheta{\hat{\theta}}
\newcommand \s{\sigma}
\newcommand \hy{\hat{y}}
\newcommand \hY{\hat{Y}}
\newcommand \e{\varepsilon}
\newcommand \he{\hat{\e}}
\newcommand \z{z}
\newcommand \hVar{\widehat{\Var}}
\newcommand \hCorr{\widehat{\Corr}}
\newcommand \hCov{\widehat{\Cov}}
\newcommand \cN{\mathcal{N}}
\newcommand \RR{\mathbb{R}}
\newcommand \NN{\mathbb{N}}
\newcommand{\cF}{\mathcal{F}}
\newcommand{\cH}{\mathcal{H}}


\begin{document}

% Кратко: очно. 

\begin{enumerate}

\item Василиса Прекрасная подбрасывает игральный кубик и записывает результат броска как $y_1$.
Далее при $t \geq 2$ она считает $y_t$ по формуле $y_t = y_{t-1} + 1$. 
Величину $x_t$ Василиса определяет как остаток от деления $y_t$ на три. 

\begin{enumerate}
    \item Является ли процесс $(x_t)$ стационарным?
    \item Постройте график теоретической автокорреляционной функции этого процесса. 
\end{enumerate}


\item Полугодовые наблюдения $(y_t)$ описываютя $ETS(ANA)$ моделью
\[
\begin{cases}
    u_t \sim \cN(0; 4) \\
    s_t = s_{t-2} + 0.1 u_t \\
    \ell_t = \ell_{t-1} + 0.3 u_t \\
    y_t = \ell_{t-1} + s_{t-2} + u_t \\
\end{cases}    
\]
Постройте 95\% предиктивный интервал для $y_{102}$, если $s_{100} = 3$, $s_{99} = -2$, $\ell_{100} = 100$.

    \item У стационарного процесса $(y_t)$ с математическим ожиданием $100$ автокорреляционная функция равна $\rho_k = 0.1^k$.
  \begin{enumerate}
    \item Найдите первые две частные автокорреляции, $\phi_{11}$ и $\phi_{22}$.
    \item Запишите возможное разностное уравнение для данного процесса. 
  \end{enumerate}
  
\item Рассмотрим разностное уравнение $y_t - 0.7y_{t-1} + 0.1y_{t-2} = u_t - 0.5u_{t-1}$, где величины $u_t$ независимы и нормально распределены $\cN(0;1)$.
\begin{enumerate}
    \item Сколько нестационарных и стационарных решений имеет это уравнение?
    \item Запишите \textit{более простое} разностное уравнение с тем же множеством стационарных решений. 
\end{enumerate}

\item Часто говорят, что у рекуррентного уравнения $y_t = y_{t-1} + u_t$ не может быть стационарного решения $(y_t)$, 
если последовательность $(u_t)$ — белый шум. 
Из этого утверждения есть одно маленькое и (подсказка!) \textit{000чень простое} исключение. 

Приведите явный пример последовательности  $(u_t)$ одинаково распределённых и независимых величин
таких, что упомянутое уравнение будет иметь бесконечное количество стационарных решений. 


\item Иван Дурак раздобыл длинный временной ряд и оценил параметры уравнения $y_t = \beta_1 + \beta_2 y_{t-1} + u_t$ 
двумя способами.
Во-первых, с помощью метода наименьших квадратов, $\hat\beta_1 = 0.3$, $\hat\beta_2 = 1.05$. 
Во-вторых, с помощью метода максимального правдоподобия, $\hat\beta_1 = 362$, $\hat\beta_2 = 0.999$,
предполагая стационарную $AR(1)$ модель, представимую в виде $MA(\infty)$.

Затем Иван переставил наблюдения в обратном порядке и повторно оценил параметры двумя способами. 
\begin{enumerate}
    \item Какие примерно результаты дал метод наименьших квадратов?
    \item Какие примерно результаты дал метод максимального правдоподобия?
\end{enumerate}


\end{enumerate}


\end{document}

