% arara: xelatex
\documentclass[12pt]{article}

\usepackage{physics}


\usepackage{tikz} % картинки в tikz
\usepackage{microtype} % свешивание пунктуации

\usepackage{array} % для столбцов фиксированной ширины

\usepackage{indentfirst} % отступ в первом параграфе

\usepackage{sectsty} % для центрирования названий частей
\allsectionsfont{\centering}

\usepackage{amsmath, amsfonts, amssymb} % куча стандартных математических плюшек

\usepackage{comment}

\usepackage[top=2cm, left=1.2cm, right=1.2cm, bottom=2cm]{geometry} % размер текста на странице

\usepackage{lastpage} % чтобы узнать номер последней страницы

\usepackage{enumitem} % дополнительные плюшки для списков
%  например \begin{enumerate}[resume] позволяет продолжить нумерацию в новом списке
\usepackage{caption}

\usepackage{url} % to use \url{link to web}

\usepackage{fancyhdr} % весёлые колонтитулы
\pagestyle{fancy}
\lhead{Анализ временных рядов}
\chead{}
\rhead{2024-05-17}
\lfoot{}
\cfoot{DON'T PANIC}
\rfoot{\thepage/\pageref{LastPage}}
\renewcommand{\headrulewidth}{0.4pt}
\renewcommand{\footrulewidth}{0.4pt}

\usepackage{tcolorbox} % рамочки!

\usepackage{todonotes} % для вставки в документ заметок о том, что осталось сделать
% \todo{Здесь надо коэффициенты исправить}
% \missingfigure{Здесь будет Последний день Помпеи}
% \listoftodos - печатает все поставленные \todo'шки


% более красивые таблицы
\usepackage{booktabs}
% заповеди из докупентации:
% 1. Не используйте вертикальные линни
% 2. Не используйте двойные линии
% 3. Единицы измерения - в шапку таблицы
% 4. Не сокращайте .1 вместо 0.1
% 5. Повторяющееся значение повторяйте, а не говорите "то же"



\usepackage{fontspec}
\usepackage{polyglossia}

\setmainlanguage{russian}
\setotherlanguages{english}

% download "Linux Libertine" fonts:
% http://www.linuxlibertine.org/index.php?id=91&L=1
\setmainfont{Linux Libertine O} % or Helvetica, Arial, Cambria
% why do we need \newfontfamily:
% http://tex.stackexchange.com/questions/91507/
\newfontfamily{\cyrillicfonttt}{Linux Libertine O}

\AddEnumerateCounter{\asbuk}{\russian@alph}{щ} % для списков с русскими буквами
\setlist[enumerate, 2]{label=\asbuk*),ref=\asbuk*}

%% эконометрические сокращения
\DeclareMathOperator{\Cov}{\mathbb{C}ov}
\DeclareMathOperator{\Corr}{\mathbb{C}orr}
\DeclareMathOperator{\Var}{\mathbb{V}ar}

\let\P\relax
\DeclareMathOperator{\P}{\mathbb{P}}

\DeclareMathOperator{\E}{\mathbb{E}}
% \DeclareMathOperator{\tr}{trace}
\DeclareMathOperator{\card}{card}
\DeclareMathOperator{\plim}{plim}
\DeclareMathOperator{\pCorr}{\mathrm{p}\mathbb{C}\mathrm{orr}}


\newcommand \hb{\hat{\beta}}
\newcommand \hs{\hat{\sigma}}
\newcommand \htheta{\hat{\theta}}
\newcommand \s{\sigma}
\newcommand \hy{\hat{y}}
\newcommand \hY{\hat{Y}}
\newcommand \e{\varepsilon}
\newcommand \he{\hat{\e}}
\newcommand \z{z}
\newcommand \hVar{\widehat{\Var}}
\newcommand \hCorr{\widehat{\Corr}}
\newcommand \hCov{\widehat{\Cov}}
\newcommand \cN{\mathcal{N}}
\newcommand \RR{\mathbb{R}}
\newcommand \NN{\mathbb{N}}
\newcommand{\cF}{\mathcal{F}}
\newcommand{\cH}{\mathcal{H}}


\begin{document}

% Кратко: очно. 

\begin{enumerate}

\item У Василисы Прекрасной есть мешочек с монетками. 
В стартовый день в мешочке находится случайное количество $y_0 \in [0;3]$ монеток. 
Каждый день с номером $t\geq 1$ Василиса выполняет ровно одно действие.  
Если в мешочке 0 монеток, то она добавляет одну монетку. 
Если в мешочке 3 монетки, то она забирает одну монетку. 
Если в мешочке 1 или 2 монетки, то она равновероятно добавляет или забирает одну монетку независимо от предыстории. 

\begin{enumerate}
    \item При каком распределении величины $y_0$ процесс $(y_t)$ будет слабо стационарным?
    \item Будет ли построенный в пункте (а) процесс сильно стационарным?
\end{enumerate}


\item Полугодовые наблюдения $(y_t)$ описываютя $ETS(ANA)$ моделью
\[
\begin{cases}
    u_t \sim \cN(0; 9) \\
    s_t = s_{t-2} + 0.1 u_t \\
    \ell_t = \ell_{t-1} + 0.4 u_t \\
    y_t = \ell_{t-1} + s_{t-2} + u_t \\
\end{cases}    
\]
Постройте 95\% предиктивный интервал для $y_{103}$, если $s_{100} = 3$, $s_{99} = -2$, $\ell_{100} = 100$.

    \item Аль Капоне обнаружил, что начиная с некоторого лага все автокорреляции случайного процесса удовлетворяют соотношению 
    $\rho_k = 0.3 \rho_{k - 1} - 0.02 \rho_{k - 2}$.
  \begin{enumerate}
    \item Приведите любой пример процесса с такой автокорреляционной функцией. 
    \item Найдите первую частную автокорреляцию $\phi_{11}$ для процесса предложенного в пункте (а).
  \end{enumerate}
  
\item Рассмотрим $MA(2)$ процесс $y_t = 5 + u_t + 3u_{t-1} + 2u_{t-2}$, где $(u_t)$ — белый шум. 

\begin{enumerate}
    \item Найдите автокорреляционную функцию процесса $(y_t)$.
    \item Приведите пример $MA(2)$ процесса $(y'_t)$ и белого шума $(u'_t)$ таких, 
    что $(y'_t)$ имеет такую же автокорреляционную функцию как $(y_t)$, но коэффициенты в формуле для $(y'_t)$
    относительно $(u'_t)$ отличаются от коэффициентов в формуле для $(y_t)$ относительно $(u_t)$.
\end{enumerate}


\item Лус дель Кармен Ибаньес Карранса из города Трухильо разыскивает все $\alpha$, при которых процессы $(u_t)$ и $(w_t)$ 
одновременно являются белыми шумами,
\[
w_t = \frac{1+\alpha L^{-1}}{1 + 0.5 L} u_t.
\]
\begin{enumerate}
\item Помогите ей найти хотя бы одно искомое $\alpha$.
\item Помогите ей найти второе $\alpha$. 
\end{enumerate}


% \item

\end{enumerate}


\end{document}

