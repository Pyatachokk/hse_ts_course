\documentclass[12pt,fleqn]{article}
\usepackage{../../vkCourseML}
%\usepackage{vkCourseML}
\hypersetup{unicode=true}
%\usepackage[a4paper]{geometry}
\usepackage[hyphenbreaks]{breakurl}

\interfootnotelinepenalty=10000
\newcommand{\dx}[1]{\,\mathrm{d}#1} % маленький отступ и прямая d

\begin{document}
\title{Моделирование временных рядов\\Семинар 2\\Стратегии прогнозирования}
\author{Борис Демешев, Матвей Зехов}
\date{}
\maketitle

\section{Подход к моделированию}

\section{Стратегии прогнозирования}

На текущий момент мы знакомы только с простейшими моделями сглаживания и сведением задачи прогнозирования к табличной форме. Последняя наиболее удобна для иллюстрации различных стратегий прогнозирования, так как любой из рассмотренных подходов можно реализовать в табличном подходе. Для специализированных моделей временных рядов стратегия обычно продиктована строением самой модели. По мере нашего экскурса мы будем отмечать, какая стратегия применима в конкретном случае. 

Допустим, мы успешно преобразовали в табличный вид. Он должен выглядеть примерно так:

\begin{table}[!h]
	\centering
	\begin{tabular}{|c|c|cccc|}
		\hline
		t & $y_t$ & $y_{t-1}$ & $y_{t-2}$ & $y_{t-3}$ & $x_{t-1}$ \\ \hline
		3 & 3     & 2         & 1         & 0         & 10        \\
		4 & 4     & 3         & 2         & 1         & 15        \\
		5 & 5     & 4         & 3         & 2         & 20        \\
		6 & 6     & 5         & 4         & 3         & 12        \\
		7 & 7     & 6         & 5         & 4         & 17        \\
		8 & 8     & 7         & 6         & 5         & 30        \\ \hline
	\end{tabular}
\end{table}
\end{document}