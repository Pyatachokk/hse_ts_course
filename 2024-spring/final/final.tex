% arara: xelatex
\documentclass[12pt]{article}

\usepackage{physics}


\usepackage{tikz} % картинки в tikz
\usepackage{microtype} % свешивание пунктуации

\usepackage{array} % для столбцов фиксированной ширины

\usepackage{indentfirst} % отступ в первом параграфе

\usepackage{sectsty} % для центрирования названий частей
\allsectionsfont{\centering}

\usepackage{amsmath, amsfonts, amssymb} % куча стандартных математических плюшек

\usepackage{comment}

\usepackage[top=2cm, left=1.2cm, right=1.2cm, bottom=2cm]{geometry} % размер текста на странице

\usepackage{lastpage} % чтобы узнать номер последней страницы

\usepackage{enumitem} % дополнительные плюшки для списков
%  например \begin{enumerate}[resume] позволяет продолжить нумерацию в новом списке
\usepackage{caption}

\usepackage{url} % to use \url{link to web}

\usepackage{fancyhdr} % весёлые колонтитулы
\pagestyle{fancy}
\lhead{Анализ временных рядов}
\chead{}
\rhead{Last dance, 2024-06-22}
\lfoot{}
\cfoot{DON'T PANIC}
\rfoot{\thepage/\pageref{LastPage}}
\renewcommand{\headrulewidth}{0.4pt}
\renewcommand{\footrulewidth}{0.4pt}

\usepackage{tcolorbox} % рамочки!

\usepackage{todonotes} % для вставки в документ заметок о том, что осталось сделать
% \todo{Здесь надо коэффициенты исправить}
% \missingfigure{Здесь будет Последний день Помпеи}
% \listoftodos - печатает все поставленные \todo'шки


% более красивые таблицы
\usepackage{booktabs}
% заповеди из докупентации:
% 1. Не используйте вертикальные линни
% 2. Не используйте двойные линии
% 3. Единицы измерения - в шапку таблицы
% 4. Не сокращайте .1 вместо 0.1
% 5. Повторяющееся значение повторяйте, а не говорите "то же"



\usepackage{fontspec}
\usepackage{polyglossia}

\setmainlanguage{russian}
\setotherlanguages{english}

% download "Linux Libertine" fonts:
% http://www.linuxlibertine.org/index.php?id=91&L=1
\setmainfont{Linux Libertine O} % or Helvetica, Arial, Cambria
% why do we need \newfontfamily:
% http://tex.stackexchange.com/questions/91507/
\newfontfamily{\cyrillicfonttt}{Linux Libertine O}

\AddEnumerateCounter{\asbuk}{\russian@alph}{щ} % для списков с русскими буквами
\setlist[enumerate, 2]{label=\asbuk*),ref=\asbuk*}

%% эконометрические сокращения
\DeclareMathOperator{\Cov}{\mathbb{C}ov}
\DeclareMathOperator{\Corr}{\mathbb{C}orr}
\DeclareMathOperator{\Var}{\mathbb{V}ar}

\let\P\relax
\DeclareMathOperator{\P}{\mathbb{P}}

\DeclareMathOperator{\E}{\mathbb{E}}
% \DeclareMathOperator{\tr}{trace}
\DeclareMathOperator{\card}{card}
\DeclareMathOperator{\plim}{plim}
\DeclareMathOperator{\pCorr}{\mathrm{p}\mathbb{C}\mathrm{orr}}


\newcommand \hb{\hat{\beta}}
\newcommand \hs{\hat{\sigma}}
\newcommand \htheta{\hat{\theta}}
\newcommand \s{\sigma}
\newcommand \hy{\hat{y}}
\newcommand \hY{\hat{Y}}
\newcommand \e{\varepsilon}
\newcommand \he{\hat{\e}}
\newcommand \z{z}
\newcommand \hVar{\widehat{\Var}}
\newcommand \hCorr{\widehat{\Corr}}
\newcommand \hCov{\widehat{\Cov}}
\newcommand \cN{\mathcal{N}}
\newcommand \RR{\mathbb{R}}
\newcommand \NN{\mathbb{N}}
\newcommand{\cF}{\mathcal{F}}
\newcommand{\cH}{\mathcal{H}}


\begin{document}

% Кратко: очно. 

\begin{enumerate}

\item Величины $(x_t)$ независимы и равновероятно принимают значения $0$ или $1$. 
Определим процесс $y_t = x_t / (1 + x_{t-1})$.

\begin{enumerate}
    \item Является ли процесс $(y_t)$ стационарным?
\end{enumerate}

Винни Пух обладает большим количеством наблюдений за $(y_t)$ и оценивает модель вида $y_t = \alpha + u_t + \beta u_{t-1}$
    методом максимального правдоподобия. 
Винни ошибочно предполагает, что величины $u_t \sim \cN(0;\sigma^2)$ и независимы.
    
\begin{enumerate}[resume]
   \item Какое примерно $\hat\beta$ получит Винни-Пух?
\end{enumerate}


\item Величины $X_1$, $X_2$, $X_3$ независимы и равномерны на $[0;1]$.
Определим пару величин $L = \min\{X_1, X_2\}$ и $R = \min\{X_1, X_2, X_3\}$.

\begin{enumerate}
    \item Постройте копулу величин $L$ и $R$. 
    \item Нарисуйте любую линию уровня этой копулы. 
\end{enumerate}


\item Стационарный в сильном смысле белый шум $(u_t)$ описывается $GARCH(1, 0)$ моделью:
$u_t = \nu_t \sigma_t$, $\nu_t \sim \cN(0;1)$, $\sigma_t^2 = 3 + 0.5 \sigma_{t-1}^2$.
Помимо уравнений предполагают, что $\nu_t$ не зависит от $u_{t-1}$, $\nu_{t-1}$, $u_{t-2}$, $\nu_{t-2}$, \ldots. 

\begin{enumerate}
    \item Найдите безусловную дисперсию $u_t$.
    \item Известно, что $\sigma_{100} = 2$. 
    Постройте 95\%-е предиктивные интервалы для $u_{101}$ и $u_{102}$.
\end{enumerate}



\item Найдите DTW-расстояние между рядами $x = (0, 1, 0)$ и $y = (1, 2, 3, 0)$. 
В качестве отличия двух отдельных наблюдений используйте $(x_i - y_j)^2$.


\item Рассмотрим байесовскую скалярную авторегрессию, $y_t = \beta y_{t-1} + u_t$ с фиксированным неслучайными $y_0 = 0$.
Для упрощения будем считать, что распределение белого шума $(u_t)$ полностью известно, $u_t \sim \cN(0; 1)$.

Априорно исследователь верит, что $\beta \sim \cN(0; 4)$.

Найдите апостериорное распределение $\beta$, если есть всего два наблюдения, $y_1 = 1$, $y_2 = 2$.


\item Перепишите стационарный $ARMA(1, 1)$ процесс $(a_t)$ с уравнением $a_t =0.3a_t + u_t + 0.5 u_{t-1}$, 
где $(u_t)$ — белый шум, в виде модели пространства состояний
\[
    \begin{cases}
        x_t = F x_{t-1} + v_t \\
        y_t = G x_t + w_t,
    \end{cases}
\]
где величины $x_0$, $(v_1, w_1)$, $(v_2, w_2)$, \ldots{ } некоррелированы. 
    
Нужно явно выписать, как устроен векторы $x_t$ и $y_t$, матрицы $F$ и $G$, как связаны шумы $v_t$ и $w_t$ с исходным шумом $u_t$.
    
    

\end{enumerate}


\end{document}

