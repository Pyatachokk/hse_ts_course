\documentclass[12pt,fleqn]{article}
\usepackage{../../vkCourseML}
%\usepackage{vkCourseML}
\hypersetup{unicode=true}
%\usepackage[a4paper]{geometry}
\usepackage[hyphenbreaks]{breakurl}

\interfootnotelinepenalty=10000
\newcommand{\dx}[1]{\,\mathrm{d}#1} % маленький отступ и прямая d

\begin{document}
\title{Моделирование временных рядов\\Лекция 1\\Введение}
\author{Борис Демешев, Матвей Зехов}
\date{}
\maketitle

Временные ряды по сути своей лишь частный случай стандартной задачи регрессии или классификации. Они возникают в большом количестве областей. Например, любая компания из ретейла будет прогнозировать количество товаров, которые необходимо поставить в магазин или даркстор. Также обычно сразу приходит на ум котировки на фондовой бирже. Из физики и инженерных задач возникает анализ  сигналов от датчиков. Введём определение временного ряда. Перед этим мы должны ввести (очень нестрого) понятие случайного процесса.

\section{Определения}

Случайный процесс -- это некоторая \emph{последовательность} случайных величин $Y_t$. Мы специально сконцентрируемся только на дискретных последовательностях с вещественными значениями, так как для большинства задач этого достаточно. Следовательно, временным рядом мы будем называть некоторую \emph{реализацию} случайного процесса, $y_t$. Иногда эту реализацию ещё называют траекторией. Также можно встретить определение, что временной ряд это и есть случайный процесс (последовательность случайных велечин), но с практической точки зрения это немного не интуитивно и мы постараемся этого избегать.

На практике обычно мы имеем дело с последовательностями с конечным числом элементов $(y_t)_{t=1}^{T}$, где T -- количество наблюдений. Иногда в учебных целях иногда будем затрагивать последовательности с бесконечным числом элементов: $(y_t)_{t=1}^{t=+\infty}$ или $(y_t)_{t=-\infty}^{t=+\infty}$

В классических моделях машинного обучения мы предполагали наблюдения в обучающей выборке независимыми и одинаково распределёнными: $X = \{(x_1, y_1), \dots, (x_\ell, y_\ell)\}$. Однако от этих предпосылок нам придётся отказаться. Почти всегда элементы последовательности будут зависимы между собой и нашей задачей будет выяснить характер этой связи. Грубо говоря, нам необходимо восстановить характеристики случайного процесса по сгенерированной тракетории. Да, наличие структуры в данных не позволяет нам напрямую использовать стандартные техники машинного обучения, но в то же время из этой структуры можно выделить много дополнительной и полезной для прогнозирования информации.


Можем ли мы в принципе быть уверены, что сможем его восстановить? В какой-то мере ответ на это даёт Теорема Дуба о разложении \cite{doob}. Говоря в очень упрощённых терминах, она говорит о том, что почти любой "хороший"\ процесс можно разложить на прогнозируемую (детерминированную) и принципиально непрогнозируемую части. Следственно, мы никогда не сможем восстановить процесс идеально. Но тем не менее, часть процессов более склонна к детерминированному поведению, а часть -- менее. Например дневная температура яввляется очень сезонной величиной, то есть имеет паттерн, удобный для прогнозирования. Мы будем учиться обнаруживать и выделять такие паттерны в данных. С другой стороны, котировки акций наиболее близки к хаотичному движению и весьма трудно поддаются прогнозированию. Этим занимаются скорее в области количественных финансов. В нашем курсе котировки акций могут быть рассмотрены скорее из-за удобства и качества данных, но не более чем для иллюстрации. Исключение составит тема прогнозирования волатильности.




\section{Возможные постановки задач}

Задачи на временных рядах можно рассматривать с двух сторон. Во-первых, их можно свести к стандартным методам машинного обучения с минимальными оговорками в подготовке данных. Во-вторых, можно рассматривать это направление как развивавшееся независимо в контексте эконометрических задач с уклоном в логику описания данных. Мы кратко поговорим про первый подход и более подробно про второй.

\subsection{Случай одного ряда}
\subsection{Случай набора рядов}

\begin{thebibliography}{1}
	\bibitem{doob}
	https://wikichi.ru/wiki/Doob\_decomposition\_theorem
\end{thebibliography}

\end{document}