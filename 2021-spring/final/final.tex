% arara: xelatex
\documentclass[12pt]{article}

\usepackage{tikz} % картинки в tikz
\usepackage{microtype} % свешивание пунктуации

\usepackage{array} % для столбцов фиксированной ширины

\usepackage{indentfirst} % отступ в первом параграфе

\usepackage{sectsty} % для центрирования названий частей
\allsectionsfont{\centering}

\usepackage{amsmath, amssymb} % куча стандартных математических плюшек

\usepackage{physics}
\usepackage{comment}

\usepackage[top=2cm, left=1.2cm, right=1.2cm, bottom=2cm]{geometry} % размер текста на странице

\usepackage{lastpage} % чтобы узнать номер последней страницы

\usepackage{enumitem} % дополнительные плюшки для списков
%  например \begin{enumerate}[resume] позволяет продолжить нумерацию в новом списке
\usepackage{caption}


\usepackage{fancyhdr} % весёлые колонтитулы
\pagestyle{fancy}
\lhead{Временные ряды}
\chead{}
\rhead{2022-06-22, праздник номер два!}
\lfoot{}
\cfoot{НЕ ПАНИКОВАТЬ}
\rfoot{\thepage/\pageref{LastPage}}
\renewcommand{\headrulewidth}{0.4pt}
\renewcommand{\footrulewidth}{0.4pt}


\let\P\relax
\DeclareMathOperator{\P}{\mathbb{P}}

\usepackage{todonotes} % для вставки в документ заметок о том, что осталось сделать
% \todo{Здесь надо коэффициенты исправить}
% \missingfigure{Здесь будет Последний день Помпеи}
% \listoftodos --- печатает все поставленные \todo'шки


% более красивые таблицы
\usepackage{booktabs}
% заповеди из докупентации:
% 1. Не используйте вертикальные линни
% 2. Не используйте двойные линии
% 3. Единицы измерения - в шапку таблицы
% 4. Не сокращайте .1 вместо 0.1
% 5. Повторяющееся значение повторяйте, а не говорите "то же"



\usepackage{fontspec}
\usepackage{polyglossia}

\setmainlanguage{russian}
\setotherlanguages{english}

% download "Linux Libertine" fonts:
% http://www.linuxlibertine.org/index.php?id=91&L=1
\setmainfont{Linux Libertine O} % or Helvetica, Arial, Cambria
% why do we need \newfontfamily:
% http://tex.stackexchange.com/questions/91507/
\newfontfamily{\cyrillicfonttt}{Linux Libertine O}

\AddEnumerateCounter{\asbuk}{\russian@alph}{щ} % для списков с русскими буквами
%\setlist[enumerate, 2]{label=\asbuk*),ref=\asbuk*}

%% эконометрические сокращения
\DeclareMathOperator{\Cov}{Cov}
\DeclareMathOperator{\Corr}{Corr}
\DeclareMathOperator{\Var}{Var}
\DeclareMathOperator{\E}{E}
\def \hb{\hat{\beta}}
\def \hs{\hat{\sigma}}
\def \htheta{\hat{\theta}}
\def \s{\sigma}
\def \hy{\hat{y}}
\def \hY{\hat{Y}}
\def \v1{\vec{1}}
\def \e{\varepsilon}
\def \he{\hat{\e}}
\def \z{z}
\def \hVar{\widehat{\Var}}
\def \hCorr{\widehat{\Corr}}
\def \hCov{\widehat{\Cov}}
\def \cN{\mathcal{N}}


\begin{document}

\begin{enumerate}

\item Рассмотрим стационарную $ARCH(1)$ модель $u_t = \sigma_t \nu_t$, где $\nu_t\sim \cN(0;1)$ и независимы,
а $\sigma_t^2= 4 + 0.5 u_{t-1}^2$.

\begin{enumerate}
	\item Постройте 95\%-й предиктивный интервал для $u_{101}$, $u_{102}$, если $u_{100} = 2$.
	\item Найдите автокорреляционную функцию процесса $u_t^2$.
\end{enumerate}


\item Посчитайте $DTW$ расстояние между рядами $a=(0, 1, 3, 1)$ и $b = (1,2,0)$. 
Отличие двух конкретных наблюдений измеряйте с помощью $\abs{a_i - b_j}$.

\item Функция $f(x)$ описывается гауссовским процессом с нулевым ожиданием $GP(0, K)$.
Задающая ковариации ядерная функция $K$ имеет вид $K(a, b) = \exp(-(a-b)^2)$. 

Постройте 95\%-й интервал для $f(1)$, если $f(0)=0$ и $f(3)=1$.

На всякий: $\exp(-1)= 0.368$, $\exp(-2)=0.135$.

\item Винни-Пух использует тест Диболда-Мариано для сравнения доходности двух стратегий добычи мёда. 
Каждый день он добывает мёд у правильных пчёл и у неправильных пчёл, обозначим $d_t$ разницу добытого количества. 

Помогите Винни-Пуху проверить гипотезу об одинаковой эффективности стратегий против гипотезы о разной эффективности. 

Известно, что стационарная $AR(1)$ модель для $d_t$ дала оценки $\hat d_t = 0.3 + 0.7 d_{t-1}$ 
с оценкой дисперсии случайной составляющей $\hat\sigma^2_u = 1$.


\item В байесовской авторегрессии априорное распределение параматра имеет вид $\beta \sim \cN(1,1)$.
Модель предполагает первое наблюдение фиксированным, а далее $y_t = \beta y_{t-1} + u_t$, где $u_t \sim \cN(0;1)$ и независимы.

Ряд короткий: $y_1 = 5$, $y_2 = 6$, $y_3 = 7$.

\begin{enumerate}
	\item Найдите апостериорное распределение $\beta$.
	\item Постройте апостериорный 95\% предиктивный интервал для $y_4$.
\end{enumerate}

\item Структура иерархического временного ряда описывается матрицей $S$. 
Мы хотим найти оптимальную матрицу согласования $G$, преобразующую вектор всех рядов в вектор рядов нижнего уровня. 
Выполнено естественное требование $SGS=S$. 

Вспомним задачу минимизации суммы всех дисперсий ошибок согласованных прогнозов,
$\trace \Var(y - \tilde{y} \mid \mathcal{F}_T) \to \min$ 	
при известной матрице $W=\Var(y - \hat y \mid \mathcal{F}_T)$.

Докажите, что она полностью эквивалентна задаче нахождения вектора $\hat \beta$ оценок с наименьшими дисперсиями вида $\hat \beta = G a$ в модели 
$a = S \beta + u$,  при  $\E(u) = 0$, $\Var(u) = W$.

Здесь $y$ — вектор всех рядов иерархии, $\hat y$ — вектор несогласованных прогнозов, $\tilde y$ — вектор согласованных прогнозов. 

\end{enumerate}

\end{document}
