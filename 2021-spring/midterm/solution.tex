\documentclass[a4paper,14pt]{article}


\usepackage[utf8]{inputenc}
\usepackage[T1,T2A]{fontenc}
\usepackage{amsmath,amsthm,amssymb}  
\usepackage{mathtext}
\usepackage[english,russian]{babel}

\usepackage[table]{xcolor}
\usepackage{ucs}
\usepackage{textcomp}
\usepackage{array}
\usepackage{indentfirst}
\usepackage{enumerate}
\usepackage{tikz}
\usetikzlibrary{shapes, arrows}
\usepackage{icomma}
\usepackage{gensymb}
\usepackage{multicol}
\usepackage{bussproofs}
\usepackage{subfiles}
\usepackage{parskip}
\usepackage{algpseudocode}
\usepackage{algorithm}
\usepackage{mathtools}
\usepackage{titling}

\setlength{\droptitle}{-8em} 

\usepackage{graphicx}
\graphicspath{ {./graphics/} }

\setlength{\parindent}{1cm}

\RequirePackage[colorlinks=true, urlcolor=blue]{hyperref}

\def\dotminus{\mathbin{\ooalign{\hss\raise1ex\hbox{.}\hss\cr\mathsurround=0pt$-$}}}
\renewcommand{\C}{\mathbb{C}}
\newcommand{\N} {\mathbb{N}}
\newcommand{\Q} {\mathbb{Q}}
\newcommand{\Z} {\mathbb{Z}}
\newcommand{\R} {\mathbb{R}}
\newcommand{\ord} {\mathop{\rm ord}}
\newcommand{\Ima}{\mathop{\rm Im}}
\newcommand{\rk}{\mathop{\rm rk}}
\newcommand{\vect}{\mathop{\rm vec}}
\newcommand{\diag}{\mathop{\rm diag}}
\newcommand{\cond}{\mspace{3mu}{|}\mspace{3mu}}

\renewcommand{\r}{\right}
\renewcommand{\l}{\left}
\newcommand{\scalar}[2]{\left\langle #1, #2 \right\rangle}
\newcommand{\Sum}[2]{\overset{#2}{\underset{#1}{\sum}}}
\newcommand{\Lim}[2]{\lim\limits_{#1 \rightarrow #2}}
\renewcommand{\baselinestretch}{1.2} % междустрочный интервал
%%% Работа с русским языком
\usepackage{cmap}					% поиск в PDF
\usepackage{mathtext} 				% русские буквы в формулах
\usepackage[T2A]{fontenc}			% кодировка
\usepackage[utf8]{inputenc}			% кодировка исходного текста
\usepackage[english,russian]{babel}	% локализация и переносы

%%% Дополнительная работа с математикой
\usepackage{amsfonts,amssymb,amsthm,mathtools} % AMS
\usepackage{amsmath}
\usepackage{icomma} % "Умная" запятая: $0,2$ --- число, $0, 2$ --- перечисление

%% Номера формул
%\mathtoolsset{showonlyrefs=true} % Показывать номера только у тех формул, на которые есть \eqref{} в тексте.
%% Шрифты
\usepackage{euscript}	 % Шрифт Евклид
\usepackage{mathrsfs} % Красивый матшрифт

% Межстрочный интервал
\usepackage{setspace}
%\полуторный интервал
\onehalfspacing
% Поля страницы
\usepackage[left=10mm, top=30mm, right=10mm, bottom=30mm, nohead, nofoot]{geometry}

%% Свои команды
\DeclareMathOperator{\sgn}{\mathop{sgn}}

%% Перенос знаков в формулах (по Львовскому)
\newcommand*{\hm}[1]{#1\nobreak\discretionary{}
	{\hbox{$\mathsurround=0pt #1$}}{}}

%%% Работа с картинками
\usepackage{graphicx}  % Для вставки рисунков
\graphicspath{ {./images/} }  % папки с картинками
\setlength\fboxsep{3pt} % Отступ рамки \fbox{} от рисунка
\setlength\fboxrule{1pt} % Толщина линий рамки \fbox{}
\usepackage{wrapfig} % Обтекание рисунков и таблиц текстом

%%% Работа с таблицами
\usepackage{array,tabularx,tabulary,booktabs} % Дополнительная работа с таблицами
\usepackage{longtable}  % Длинные таблицы
\usepackage{multirow} % Слияние строк в таблице


\usepackage{indentfirst}
\usepackage{hyperref}
\usepackage{booktabs}
\usepackage{float}
\usepackage[table]{xcolor}

% код в matlab
\usepackage{matlab-prettifier}
\usepackage{listings,lstautogobble}
\usepackage[russian]{babel}
\lstset{language=R, extendedchars=\true}
\lstset{autogobble=true}

% красивая рамочка
\usepackage{tcolorbox} 

\DeclareMathOperator{\Cov}{Cov}
\DeclareMathOperator{\Corr}{Corr}
\DeclareMathOperator{\Var}{Var}
\DeclareMathOperator{\E}{E}
\def \hb{\hat{\beta}}
\def \hs{\hat{\sigma}}
\def \htheta{\hat{\theta}}
\def \s{\sigma}
\def \hy{\hat{y}}
\def \hY{\hat{Y}}
\def \v1{\vec{1}}
\def \e{\varepsilon}
\def \he{\hat{\e}}
\def \z{z}
\def \hVar{\widehat{\Var}}
\def \hCorr{\widehat{\Corr}}
\def \hCov{\widehat{\Cov}}
\def \cN{\mathcal{N}}

\usepackage{etaremune}

\title{TS midterm 2022}
\date{}
% \pagenumbering{gobble}
\usepackage{layout}


\begin{document}

\maketitle

\begin{enumerate}

\item Известно, что $(u_t)$ — белый шум, а $(y_t)$ равен 
\[
y_t = \frac{1 + 3L}{1 - 0.2 L} (5 + u_t).	
\]
\begin{enumerate}
	\item Найдите $\E(y_t)$, $\Var(y_t)$, $\Cov(y_t, y_s)$.
	\item Стационарен ли процесс $(y_t)$?
	\item Запишите рекуррентное уравнение на $y_t$, $u_t$ и их лаги, решением которого является данный процесс. 
\end{enumerate}

{\bf Решение:}

Проще всего выполнять пункты в обратном порядке:
\begin{etaremune}
	\item        

 \[\begin{aligned}
y_t = \frac{1 + 3L}{1 - 0.2 L} (5 + u_t) 
	\end{aligned}\]
  \[\begin{aligned}
y_t - 0.2y_{t-1} = (1 + 3L) (5 + u_t) 
	\end{aligned}\]
   \[\begin{aligned}
y_t - 0.2y_{t-1} = 20 +  u_t + 3u_{t-1}
	\end{aligned}\]
 \[\begin{aligned}
y_t = 0.2y_{t-1} +  u_t + 3u_{t-1} + 20
	\end{aligned}\]

 
	\item Перед нами ARMA(1,1). MA-часть стационарна по определению. Проверим стационарность AR-части с помощью корней лагового многочлена:
 \[\begin{aligned}
1-0.2L = 0
	\end{aligned}\]
  \[\begin{aligned}
L = 5
	\end{aligned}\]
Единственный корень $L=5 > 1$, следовательно, процесс стационарен
 
	\item \[\begin{aligned}
		\E(y_t) = \E(\frac{1 + 3L}{1 - 0.2 L} (5 + u_t)) = \E(\frac{5 + 15L}{1 - 0.2 L} + u_t\frac{1 + 3L}{1 - 0.2 L}) =  \\ = \E(\frac{5 + 15L}{1 - 0.2 L}) + \frac{1 + 3L}{1 - 0.2 L}\E(u_t)  = \E(\frac{5 + 15}{1 - 0.2}) + 0 = \frac{5 + 15}{1 - 0.2} = 25
	\end{aligned}\]

Заметим, что $\Var(y_t) = \Var(y_{t-1})$, так как процесс стационарен, поэтому:

  \[\begin{aligned}
\Var(y_t) = \Var(0.2y_{t-1} +  u_t + 3u_{t-1} + 20)
	\end{aligned}\]
   \[\begin{aligned}
\Var(y_t) = 0.04\Var(y_{t}) +  \s^2 + 9\s^2 + 2\Cov(0.2y_{t-1}, u_t) + 2\Cov(0.2y_{t-1}, 3u_{t-1}) + 2\Cov(u_t, u_{t-1})
	\end{aligned}\]
    \[\begin{aligned}
0.96\Var(y_t) = 10\s^2 + 0 + 1.2\Cov(0.2y_{t-2} + u_{t-1} + 3u_{t-2}, u_{t-1}) + 0
	\end{aligned}\]
    \[\begin{aligned}
\Var(y_t) = \frac{10\s^2 + 1.2\s^2}{0.96} = \frac{35\s^2}{3}
	\end{aligned}\]

 

Теперь посчитаем ковариацию в общем виде:


$$
\Cov(y_t, y_s) = \Cov(0.2y_{t-1} +  u_t + 3u_{t-1}, 0.2y_{s-1} +  u_s + 3u_{s-1}) = 0.04\Cov(y_{t-1}, y_{s-1}) + 0.2\Cov(y_{t-1}, u_{s}) + $$
$$+ 0.6\Cov(y_{t-1}, u_{s-1})  +0.2\Cov(y_{s-1}, u_{t}) + \Cov(u_{t}, u_{s}) + 3\Cov(u_{t}, u_{s-1}) + 0.6\Cov(y_{s-1}, u_{t-1}) 
$$
$$+ 3\Cov(u_{t-1}, u_{s}) + 9\Cov(u_{t-1}, u_{s-1})$$

 Можно заметить, что в соответствии с данной формулой ACF сводится к следующему виду:
\[
\begin{cases}
\gamma_1 = 0.04\gamma_1 + 0.2\s^2 + 1.8\s^2 + 3\s^2 \\
\gamma_2 = 0.04\gamma_1  \\
\gamma_3 = 0.04^2\gamma_1 \\
\gamma_4 = 0.04^3\gamma_1 \\
... \\
\end{cases}
\]
\[
\begin{cases}
\gamma_1 = \frac{5}{0.96}\s^2 \\
\gamma_t = 0.04^{t-1}\gamma_1, \quad t > 1  \\
\end{cases}
\]
 
\end{etaremune}

\item Рассмотрим уравнение $y_t = 3 + 0.5 y_{t-1} - 0.06 y_{t-2} + u_t - 0.2 u_{t-1}$, где $(u_t)$ — белый шум. 

\begin{enumerate}
	\item Запишите уравнение с помощью лаговых полиномов и разложите полиномы на сомножители. 
	\item Присмотревшись пристальным взглядом к корням явно выпишите хотя бы одно стационарное решение этого уравнения. 
	Является ли стационарное решение единственным?
	\item Найдите $\Corr(y_t, y_{t-k})$ для всех стационарных решений. 
\end{enumerate}

{\bf Решение:}

\begin{enumerate}
	\item $$y_t = 3 + 0.5 y_{t-1} - 0.06 y_{t-2} + u_t - 0.2 u_{t-1}$$
     \[\begin{aligned}
(1-0.5L+0.06L^2)y_t = 3 + (1-0.2L)u_t
	\end{aligned}\]
      \[\begin{aligned}
0.06(L-5)(L-\frac{10}{3})y_t = 3 + (1-0.2L)u_t
	\end{aligned}\]
       \[\begin{aligned}
y_t = \frac{3}{0.06(L-5)(L-\frac{10}{3})} + \frac{(1-0.2L)u_t}{ 0.06(L-5)(L-\frac{10}{3})} 
	\end{aligned}\]
       \[\begin{aligned}
y_t = \frac{3}{0.56} + \frac{(1-0.2L)u_t}{ 0.06(L-5)(L-\frac{10}{3})} 
	\end{aligned}\]



 
	\item Оба корня лагового многочлена больше 1, следовательно, решение стационарно и единственно

 Заметим, что:
       \[\begin{aligned}
1-0.2L =  \frac{-5 + L}{-5} 
	\end{aligned}\]

        \[\begin{aligned}
y_t = \frac{3}{0.56} + \frac{u_t}{1-0.3L} 
	\end{aligned}\]

 Согласно формуле для суммы бесконечно убывающей геометрической прогрессии:
         \[\begin{aligned}
\frac{1}{1-0.3L}  = 1 + 0.3L + (0.3L)^2 + (0.3L)^2 + ...
	\end{aligned}\]

         \[\begin{aligned}
y_t = \frac{3}{0.56} + u_t + 0.3u_{t-1} + 0.09u_{t-2} + 0.027u_{t-3} + ...
	\end{aligned}\]
 
	\item 

          \[\begin{aligned}
\Corr(y_{t}, y_{t-k}) = \frac{\gamma_k}{\gamma_0} 
	\end{aligned}\]

        \[\begin{aligned}
y_t = \frac{3}{0.56} + \frac{u_t}{1-0.3L} 
	\end{aligned}\]
         \[\begin{aligned}
y_t(1-0.3L) = 3.75 + u_t
	\end{aligned}\]
          \[\begin{aligned}
y_t = 0.3y_{t-1} + u_t + 3.75
	\end{aligned}\]
           \[\begin{aligned}
\gamma_0 = \Var(y_t) = \Var(0.3y_{t-1} + u_t) = 0.09\Var(y_t) + \s^2 + 0
	\end{aligned}\]
           \[\begin{aligned}
\gamma_0 = \frac{\s^2}{0.91}
	\end{aligned}\]
            \[\begin{aligned}
\gamma_k = \Cov(y_t, y_{t-k}) = \Cov(0.3^ky_{t-k} + 0.3^{k-1}u_{t-k+1} + 0.3^{k-2}u_{t-k+2} + ... +u_t , y_{t-k})  = 0.3^k\gamma_0 + 0
	\end{aligned}\]
            \[\begin{aligned}
\gamma_k = \Cov(y_t, y_{t-k}) = \Cov(0.3^ky_{t-k} + 0.3^{k-1}u_{t-k+1} + 0.3^{k-2}u_{t-k+2} + ... +u_t , y_{t-k})  = \s^2\frac{0.3^k}{0.91}
	\end{aligned}\]
 
\end{enumerate}



\item Вспомним $ETS(AAN)$ модель, которая описывается системой уравнений

\[
\begin{cases}
y_t = \ell_{t-1} + b_{t-1} + u_t \\
\ell_t = \ell_{t-1} + b_{t-1} + \alpha u_t \\
b_t = b_{t-1} + \beta u_t \\
u_t \sim \cN(0;\sigma^2). \\
% s_t = s_{t-12} + \gamma \varepsilon_t \\
\end{cases}
\]

\begin{enumerate}
	\item Выпишите список параметров данной модели и логарифм функции плотности $y_2$ через выписанные параметры. 
	
	\item Для $l_{100} = 30$, $b_{100} = 1$, $\alpha=0.2$, $\beta=0.3$, $\sigma^2 = 16$ постройте
	интервальный прогноз на один и два шага вперёд. 
\end{enumerate}

{\bf Решение:}

\begin{enumerate}
	\item Параметры модели: $b_0, l_0, \alpha, \beta, \s^2$

             \[\begin{aligned}
y_2 = l_0 + b_0 + \alpha u_1 + b_0 + \beta u_1 + u_2
	\end{aligned}\]
              \[\begin{aligned}
y_2 = l_0 + 2b_0 + (\alpha + \beta) u_1 + u_2
	\end{aligned}\]

               \[\begin{aligned}
\E(y_2) = l_0 + 2b_0
	\end{aligned}\]
 
                \[\begin{aligned}
\Var(y_2) = (\alpha + \beta)^2*\s^2 + s^2 = \s^2(1 + (\alpha + \beta)^2)
	\end{aligned}\]
                 \[\begin{aligned}
\cN(\mu;\sigma^2) = \frac{1}{\sqrt{2\pi \s^2}} \exp{\frac{-(x-\mu)^2}{2\s^2}}
	\end{aligned}\]
                 \[\begin{aligned}
f_{y_2}(x) = \frac{1}{\sqrt{2\pi \s^2(1 + (\alpha + \beta)^2)}} \exp{\frac{-(x-(l_0 + 2b_0))^2}{2\s^2(1 + (\alpha + \beta)^2)}} 
	\end{aligned}\]
                  \[\begin{aligned}
log(f_{y_2}(x)) = - \frac{1}{2}log(2\pi \s^2(1 + (\alpha + \beta)^2)) - \frac{(x-(l_0 + 2b_0))^2}{2\s^2(1 + (\alpha + \beta)^2)}
	\end{aligned}\]
 
	\item    
 
 \[\begin{aligned}
PCI(95\%) = [ \hat{y} - 1.96\sqrt{\Var(\hat{y})}; \hat{y} + 1.96\sqrt{\Var(\hat{y})}]
	\end{aligned}\]

  \[\begin{aligned}
y_{101} = l_{100} + b_{100} + u_{101} = 31 + u_{101} 
	\end{aligned}\]
   \[\begin{aligned}
\E(y_{101}|I_{100}) = 31 
	\end{aligned}\]
    \[\begin{aligned}
\Var(y_{101}|I_{100}) = \s^2 = 16
	\end{aligned}\]
  \[\begin{aligned}
PCI(95\%) = [ 31 - 1.96*4; 31+ 1.96*4]
	\end{aligned}\]
   \[\begin{aligned}
PCI(95\%) = [ 23.16; 38.84]
	\end{aligned}\]

   \[\begin{aligned}
y_{102} = l_{101} + b_{101} + u_{102} = l_{100} + b_{100} + \alpha u_{101} + b_{100} + \beta u_{101} + u_{102}
	\end{aligned}\]
    \[\begin{aligned}
y_{102} = 30+1+0.2 u_{101} + 1 + 0.3 u_{101} + u_{102} = 32 + 0.5u_{101} + u_{102}
	\end{aligned}\]
   \[\begin{aligned}
\E(y_{102}|I_{100}) = 32 
	\end{aligned}\]
    \[\begin{aligned}
\Var(y_{102}|I_{100}) = 0.25\s^2 + \s^2 = 20
	\end{aligned}\]
  \[\begin{aligned}
PCI(95\%) = [ 32 - 1.96\sqrt{20}; 32+ 1.96\sqrt{20}]
	\end{aligned}\]
   \[\begin{aligned}
PCI(95\%) \approx [ 22.2; 39.8]
	\end{aligned}\]

 
\end{enumerate}


\item Приведите пример стационарного процесса, 
у которого все частные коррелляции равны нулю \textit{кроме} частной коррелляции тринадцатого порядка.
Либо докажите, что такой процесс не существует. 

{\bf Решение:}

Кажется, что самый очевидный пример - это AR(13), где единственный ненулевой коэффициент стоит перед 13 лагом. Как известно, у AR(p) процесса частные автокорреляции равны 0 после p-й частной автокорреляции. Если же сделать все коэффициенты, кроме одного, равными нулю, то останется лишь одна ненулевая частная автокорреляция при единственном ненулевом коэффициенте. Единственное условие, которое нужно выполнить - это стационарность. AR(p) процесс стационарен, если корни лагового многочлена больше единицы, следовательно:

  \[\begin{aligned}
y_{t} = \alpha + \beta_{13}*y_{t-13} + \epsilon_t
	\end{aligned}\]
   \[\begin{aligned}
(1-\beta_{13}L^{13}) = 0
	\end{aligned}\]
    \[\begin{aligned}
L =  \sqrt[13]{\frac{1}{\beta_{13}}} > 1
	\end{aligned}\]
Например, пусть $\beta_{13} = 0.2$ и для определённости $\alpha = 0.5$, тогда мы имеем стационарный процесс, у которого все частные автокорреляции равны нулю, кроме частной автокорреляции 13 порядка:
  \[\begin{aligned}
y_{t} = 0.5 + 0.2*y_{t-13} + \epsilon_t
	\end{aligned}\]

\item Величины $x_t$ независимы и равновероятно принимают значения $0$ или $1$ каждая. 
Рассмотрим процесс $r_t = x_t \cdot x_{t-1} - 0.25$.
\begin{enumerate}
	\item Стационарен ли процесс $(r_t)$?
	\item Илон Маск утверждает, что это типичный $MA(1)$ процесс, а потому он представим в виде $r_t = u_t + \alpha u_{t-1}$.
	
	Прав ли Илон Маск? Если прав, то явно запишите $u_t$ через $(x_t)$ и константы. 
\end{enumerate}

{\bf Решение:}

\begin{enumerate}
	\item Процесс стационарен, если его математическое ожидание, дисперсия и ковариация не зависят от t.

 Описанные случайные величины по сути принадлежат к распределению Бернулли с p = 0.5. 

 Исходя из того, что обе величины бинарные, их произведение также принимает либо значение 1, либо значение 0, но уже с другими вероятностями. Отметим также, что возведение в квадрат не меняет значения бинарных величин. Построим соответствующие таблицы:

 \[
\begin{array}{w{c}{8mm}|w{c}{8mm}}
x_t & P(x_t) \\
\hline
0 & 1/2 \\
1 & 1/2 
\end{array}
\qquad
\begin{array}{c|c}
x_{t-1} & P(x_{t-1}) \\
\hline
0 & 1/2 \\
1 & 1/2 
\end{array}
\qquad
\begin{array}{c|c}
x_t*x_{t-1} & P(x_t*x_{t-1}) \\
\hline
0 & 3/4 \\
1 & 1/4 
\end{array}
\]

  \[\begin{aligned}
\E(r_t) = \E(x_t * x_{t-1} - 0.25) = \E(x_t * x_{t-1}) - 0.25 = 0*\frac{3}{4} + 1*\frac{1}{4} - 0.25 = 0
	\end{aligned}\]
  \[\begin{aligned}
\Var(r_t) = \Var(x_t * x_{t-1} - 0.25) = \Var(x_t * x_{t-1}) = \E((x_t * x_{t-1})^2) - \E(x_t * x_{t-1})^2
	\end{aligned}\]
 \[\begin{aligned}
\Var(r_t) = \E(x_t^2 * x_{t-1}^2) - \E(x_t * x_{t-1})^2 = \frac{1}{4} - \frac{1}{16} = \frac{3}{16}
	\end{aligned}\]
  \[\begin{aligned}
\Cov(r_t, r_{t-k}) = \Cov(x_t * x_{t-1}, x_{t-k} * x_{t-1-k}) = \E(x_t * x_{t-1} * x_{t-k} * x_{t-1-k}) - \E(x_t * x_{t-1})\E( x_{t-k} * x_{t-1-k})
	\end{aligned}\]
\[
\begin{cases}
\gamma_1 = \frac{1}{8} - \frac{1}{16} = \frac{1}{16} \\
\gamma_k = \frac{1}{16} - \frac{1}{16} = 0, \quad k > 1  \\
\end{cases}
\]

 Легко заметить, что ни математическое ожидание, ни дисперсия, ни ковариация не зависят от момента времени, следовательно, процесс стационарен.
 
	\item Илон Маск прав. Можно заметить, что функция автокорреляции обрывается после первого лага, что соответствует MA(1). Представим шум $u_t$ через $(x_t)$ и константы.

  \[\begin{aligned}
r_t = u_t + \alpha u_{t-1}
	\end{aligned}\]
   \[\begin{aligned}
\Var(r_t) = \Var(u_t + \alpha u_{t-1}) = \s^2 + \alpha^2\s^2 = s^2(1+\alpha^2)
	\end{aligned}\]
    \[\begin{aligned}
\Cov(r_t, r_{t-1}) = \Cov(u_t + \alpha u_{t-1}, u_{t-1} + \alpha u_{t-2}) = \alpha \s^2 
	\end{aligned}\]
\[
\begin{cases}
s^2(1+\alpha^2) = \frac{3}{16} \\
\alpha \s^2  = \frac{1}{16} \\
\end{cases}
\]
    \[\begin{aligned}
\frac{s^2(1+\alpha^2)}{\alpha \s^2} = \frac{1+\alpha^2}{\alpha}  = 3
	\end{aligned}\]
    \[\begin{aligned}
\alpha_1 = \frac{3+\sqrt{5}}{2} \approx 2.6
	\end{aligned}\]
     \[\begin{aligned}
\alpha_2 = \frac{3-\sqrt{5}}{2} \approx 0.4
	\end{aligned}\]
   \[\begin{aligned}
r_t = u_t + \alpha u_{t-1}
	\end{aligned}\]
    \[\begin{aligned}
x_t * x_{t-1} - 0.25 = u_t(1 + \alpha L)
	\end{aligned}\]
     \[\begin{aligned}
 u_t = \frac{x_t * x_{t-1}}{1 + \alpha L} - \frac{0.25}{1 + \alpha}
	\end{aligned}\]
 Заметим, что есть возможность воспользоваться формулой суммы бесконечно убывающей прогрессии, поэтому возьмём $\alpha \approx 0.4$:
      \[\begin{aligned}
 u_t = \frac{x_t * x_{t-1}}{1 + 0.4 L} - 0.2
	\end{aligned}\]
       \[\begin{aligned}
 u_t = - 0.2 + x_t * x_{t-1} - 0.4x_{t-1} * x_{t-2} + 0.4^2x_{t-2} * x_{t-3} - 0.4^3x_{t-3} * x_{t-4} + ... 
	\end{aligned}\]
 
\end{enumerate}

\end{enumerate}







\end{document}
