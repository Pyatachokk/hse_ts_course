\documentclass[12pt]{article}

\usepackage{tikz} % картинки в tikz
\usepackage{microtype} % свешивание пунктуации

\usepackage{array} % для столбцов фиксированной ширины

\usepackage{indentfirst} % отступ в первом параграфе

% arara: xelatex
\usepackage{sectsty} % для центрирования названий частей
\allsectionsfont{\centering}

\usepackage{amsmath, amssymb} % куча стандартных математических плюшек

\usepackage{physics}
\usepackage{comment}

\usepackage[top=2cm, left=1.2cm, right=1.2cm, bottom=2cm]{geometry} % размер текста на странице

\usepackage{lastpage} % чтобы узнать номер последней страницы

\usepackage{enumitem} % дополнительные плюшки для списков
%  например \begin{enumerate}[resume] позволяет продолжить нумерацию в новом списке
\usepackage{caption}


\usepackage{fancyhdr} % весёлые колонтитулы
\pagestyle{fancy}
\lhead{Временные ряды}
\chead{}
\rhead{2022-03-14, праздник номер раз!}
\lfoot{}
\cfoot{НЕ ПАНИКОВАТЬ}
\rfoot{\thepage/\pageref{LastPage}}
\renewcommand{\headrulewidth}{0.4pt}
\renewcommand{\footrulewidth}{0.4pt}


\let\P\relax
\DeclareMathOperator{\P}{\mathbb{P}}

\usepackage{todonotes} % для вставки в документ заметок о том, что осталось сделать
% \todo{Здесь надо коэффициенты исправить}
% \missingfigure{Здесь будет Последний день Помпеи}
% \listoftodos --- печатает все поставленные \todo'шки


% более красивые таблицы
\usepackage{booktabs}
% заповеди из докупентации:
% 1. Не используйте вертикальные линни
% 2. Не используйте двойные линии
% 3. Единицы измерения - в шапку таблицы
% 4. Не сокращайте .1 вместо 0.1
% 5. Повторяющееся значение повторяйте, а не говорите "то же"



\usepackage{fontspec}
\usepackage{polyglossia}

\setmainlanguage{russian}
\setotherlanguages{english}

% download "Linux Libertine" fonts:
% http://www.linuxlibertine.org/index.php?id=91&L=1
\setmainfont{Linux Libertine O} % or Helvetica, Arial, Cambria
% why do we need \newfontfamily:
% http://tex.stackexchange.com/questions/91507/
\newfontfamily{\cyrillicfonttt}{Linux Libertine O}

\AddEnumerateCounter{\asbuk}{\russian@alph}{щ} % для списков с русскими буквами
%\setlist[enumerate, 2]{label=\asbuk*),ref=\asbuk*}

%% эконометрические сокращения
\DeclareMathOperator{\Cov}{Cov}
\DeclareMathOperator{\Corr}{Corr}
\DeclareMathOperator{\Var}{Var}
\DeclareMathOperator{\E}{E}
\def \hb{\hat{\beta}}
\def \hs{\hat{\sigma}}
\def \htheta{\hat{\theta}}
\def \s{\sigma}
\def \hy{\hat{y}}
\def \hY{\hat{Y}}
\def \v1{\vec{1}}
\def \e{\varepsilon}
\def \he{\hat{\e}}
\def \z{z}
\def \hVar{\widehat{\Var}}
\def \hCorr{\widehat{\Corr}}
\def \hCov{\widehat{\Cov}}
\def \cN{\mathcal{N}}


\begin{document}

Дорогие Орлицы и Орлы! Удачи вам на первом празднике по временным рядам!
Начните с того, что напишите клятву и подпишитесь под ней:

\vspace{10pt}
\textit{Я клянусь честью студента, что буду выполнять эту работу самостоятельно.}
\vspace{10pt}


А теперь — задачки:

\begin{enumerate}

\item Известно, что $(u_t)$ — белый шум, а $(y_t)$ равен 
\[
y_t = \frac{1 + 3L}{1 - 0.2 L} (5 + u_t).	
\]
\begin{enumerate}
	\item Найдите $\E(y_t)$, $\Var(y_t)$, $\Cov(y_t, y_s)$.
	\item Стационарен ли процесс $(y_t)$?
	\item Запишите рекуррентное уравнение на $y_t$, $u_t$ и их лаги, решением которого является данный процесс. 
\end{enumerate}

\item Рассмотрим уравнение $y_t = 3 + 0.5 y_{t-1} - 0.06 y_{t-2} + u_t - 0.2 u_{t-1}$, где $(u_t)$ — белый шум. 

\begin{enumerate}
	\item Запишите уравнение с помощью лаговых полиномов и разложите полиномы на сомножители. 
	\item Присмотревшись пристальным взглядом к корням явно выпишите хотя бы одно стационарное решение этого уравнения. 
	Является ли стационарное решение единственным?
	\item Найдите $\Corr(y_t, y_{t-k})$ для всех стационарных решений. 
\end{enumerate}

\item Вспомним $ETS(AAN)$ модель, которая описывается системой уравнений

\[
\begin{cases}
y_t = \ell_{t-1} + b_{t-1} + u_t \\
\ell_t = \ell_{t-1} + b_{t-1} + \alpha u_t \\
b_t = b_{t-1} + \beta u_t \\
u_t \sim \cN(0;\sigma^2). \\
% s_t = s_{t-12} + \gamma \varepsilon_t \\
\end{cases}
\]

\begin{enumerate}
	\item Выпишите список параметров данной модели и логарифм функции плотности $y_2$ через выписанные параметры. 
	
	\item Для $l_{100} = 30$, $b_{100} = 1$, $\alpha=0.2$, $\beta=0.3$, $\sigma^2 = 16$ постройте
	интервальный прогноз на один и два шага вперёд. 
\end{enumerate}


\item Приведите пример стационарного процесса, 
у которого все частные коррелляции равны нулю \textit{кроме} частной коррелляции тринадцатого порядка.
Либо докажите, что такой процесс не существует. 


\item Величины $x_t$ независимы и равновероятно принимают значения $0$ или $1$ каждая. 
Рассмотрим процесс $r_t = x_t \cdot x_{t-1} - 0.25$.
\begin{enumerate}
	\item Стационарен ли процесс $(r_t)$?
	\item Илон Маск утверждает, что это типичный $MA(1)$ процесс, а потому он представим в виде $r_t = u_t + \alpha u_{t-1}$.
	
	Прав ли Илон Маск? Если прав, то явно запишите $u_t$ через $(x_t)$ и константы. 
\end{enumerate}

\item Приведите пример стационарного процесса $(y_t)$ такого, что $\abs{\Corr(y_t, y_s)} < 1$, 
однако $\sup_{s, t} \abs{\Corr(y_t, y_s)} = 1$. Либо докажите, что данный процесс не существует. 


\end{enumerate}

\end{document}
