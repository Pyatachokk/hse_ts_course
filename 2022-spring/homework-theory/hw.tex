\documentclass[11pt,fleqn]{article}
\usepackage{vkCourseML}

%% эконометрические сокращения
\def \hb{\hat{\beta}}
\DeclareMathOperator{\sVar}{sVar}
\DeclareMathOperator{\sCov}{sCov}
\DeclareMathOperator{\sCorr}{sCorr}


\def \hs{\hat{s}}
\def \hy{\hat{y}}
\def \hY{\hat{Y}}
\def \he{\hat{\varepsilon}}
\def \v1{\vec{1}}
\def \cN{\mathcal{N}}
\def \e{\varepsilon}
\def \z{z}

\def \hVar{\widehat{\Var}}
\def \hCorr{\widehat{\Corr}}
\def \hCov{\widehat{\Cov}}

\DeclareMathOperator{\tr}{tr}
\DeclareMathOperator*{\plim}{plim}


%% лаг
\renewcommand{\L}{\mathrm{L}}


\usepackage{pgf}
\usepackage{tikz}
\usetikzlibrary{arrows,automata}


% DEFS
\def \mbf{\mathbf}
\def \msf{\mathsf}
\def \mbb{\mathbb}
\def \tbf{\textbf}
\def \tsf{\textsf}
\def \ttt{\texttt}
\def \tbb{\textbb}

\def \wh{\widehat}
\def \wt{\widetilde}
\def \ni{\noindent}
\def \ol{\overline}
\def \cd{\cdot}
\def \bl{\bigl}
\def \br{\bigr}
\def \Bl{\Bigl}
\def \Br{\Bigr}
\def \fr{\frac}
\def \bs{\backslash}
\def \lims{\limits}
\def \arg{{\operatorname{arg}}}
\def \dist{{\operatorname{dist}}}
\def \VC{{\operatorname{VCdim}}}
\def \card{{\operatorname{card}}}
\def \sgn{{\operatorname{sign}\,}}
\def \sign{{\operatorname{sign}\,}}
\def \xfs{(x_1,\ldots,x_{n-1})}
\def \Tr{{\operatorname{\mbf{Tr}}}}
\DeclareMathOperator*{\amn}{arg\,min}
\DeclareMathOperator*{\amx}{arg\,max}
\def \cov{{\operatorname{Cov}}}
\DeclareMathOperator{\Cov}{Cov}
\DeclareMathOperator{\Corr}{Corr}

\def \xfs{(x_1,\ldots,x_{n-1})}
\def \ti{\tilde}
\def \wti{\widetilde}


\def \mL{\mathcal{L}}
\def \mW{\mathcal{W}}
\def \mH{\mathcal{H}}
\def \mC{\mathcal{C}}
\def \mE{\mathcal{E}}
\def \mN{\mathcal{N}}
\def \mA{\mathcal{A}}
\def \mB{\mathcal{B}}
\def \mU{\mathcal{U}}
\def \mV{\mathcal{V}}
\def \mF{\mathcal{F}}

\def \R{\mbb R}
\def \N{\mbb N}
\def \Z{\mbb Z}
\def \P{\mbb{P}}
%\def \p{\mbb{P}}
\def \E{\mbb{E}}
\def \F{\mbb{F}}
\def \D{\msf{D}}
\def \I{\mbf{I}}
\def \L{\mathcal{L}}

\def \a{\alpha}
\def \b{\beta}
\def \t{\tau}
\def \dt{\delta}
\def \e{\varepsilon}
\def \ga{\gamma}
\def \kp{\varkappa}
\def \la{\lambda}
\def \sg{\sigma}
\def \sgm{\sigma}
\def \tt{\theta}
\def \ve{\varepsilon}
\def \Dt{\Delta}
\def \La{\Lambda}
\def \Sgm{\Sigma}
\def \Sg{\Sigma}
\def \Tt{\Theta}
\def \Om{\Omega}
\def \om{\omega}

\theorembodyfont{\rmfamily}
\newtheorem{esProblem}{Задача}

\begin{document}
\pagenumbering{gobble}
\title{Моделирование временных рядов\\Теоретическое ДЗ}
\date{}
\author{}
\maketitle

\begin{esProblem}[2 балла]
	.
	
	Пусть дан следующий процесс: 
	
	$$ ARMA(1,1): y_{t}=5+0.3 y_{t-1}+0.4 \varepsilon_{t-1}+\varepsilon_{t} $$
	
	Найдите:
	\begin{enumerate}
	\item 	$\E(y_t)$ (0.5 балла)
	\item Найдите первые три значения автокорреляционной функции $\rho_{1}, \rho_{2}, \rho_{3}$. (0.5 балла)
	\item Найдите первые три значения частной автокорреляционной функции $\phi_{11}, \phi_{22}, \phi_{33}$. (1 балл)
\end{enumerate}
\end{esProblem}

\begin{esProblem}[2 балла]
	Рассмотрим ETS-AAN модель с $\alpha=1 / 2, \beta=3 / 4, l_{99}=8, b_{99}=1, y_{99}=10, y_{100}=8, \sigma^{2}=16$. Распределение случайной ошибки считайте нормальным.
	
	\begin{enumerate}
		\item Найдите $l_{100}, b_{100}, l_{98}, b_{98}$ (0.5 балла)
		\item Постройте точечный прогноз $\hat{y}_{101 \mid 100}, \hat{y}_{102 \mid 100}$ (0.5 балла)
		\item  Постройте 95\%-ый предиктивный интервал для $y_{101}$ и $y_{102}$. (1 балл)
	\end{enumerate}
\end{esProblem}

\begin{esProblem}[3 балла] Монетка выпадает орлом с вероятностью $\cos^2 \alpha$ и решкой с вероятностью $\sin^2 \alpha$. 

\begin{enumerate}
	\item Найдите информацию Фишера об $\alpha$, содержащуюся в одном броске монетки. (1 балл)
	\item Найдите априорное распределение Джеффриса на параметр $\alpha$. (1 балл)
	\item Найдите апостериорное распределение $\alpha$, если монетка из двух бросков оба раза выпала орлом, а 
	в качестве априорного распределения было использовано распределение Джеффриса. (1 балл)
\end{enumerate}
\end{esProblem}


\begin{esProblem}[3 балла]
 Величины $X_1$, $X_2$, $X_3$ распределены независимо и равномерно на отрезке $[0;1]$. 
 Рассмотрим $L = \min \{X_1, X_2\}$, $R = \max\{X_2, X_3\}$. 

\begin{enumerate}
\item Выведите копулу для $C$ величин $L$ и $R$. (2 балла)
\item С помощью симуляций постройте диаграмму рассеяния, соответствующую данной копуле. (1 балл)
\end{enumerate}
\end{esProblem}



\newpage


\end{document}
