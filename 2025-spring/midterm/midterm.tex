% arara: xelatex
\documentclass[12pt]{article}

\usepackage{physics}


\usepackage{tikz} % картинки в tikz
\usepackage{microtype} % свешивание пунктуации

\usepackage{array} % для столбцов фиксированной ширины

\usepackage{indentfirst} % отступ в первом параграфе

\usepackage{sectsty} % для центрирования названий частей
\allsectionsfont{\centering}

\usepackage{amsmath, amsfonts, amssymb} % куча стандартных математических плюшек

\usepackage{comment}

\usepackage[top=2cm, left=1.2cm, right=1.2cm, bottom=2cm]{geometry} % размер текста на странице

\usepackage{lastpage} % чтобы узнать номер последней страницы

\usepackage{enumitem} % дополнительные плюшки для списков
%  например \begin{enumerate}[resume] позволяет продолжить нумерацию в новом списке
\usepackage{caption}

\usepackage{url} % to use \url{link to web}

\usepackage{fancyhdr} % весёлые колонтитулы
\pagestyle{fancy}
\lhead{Моделирование временных рядов}
\chead{}
\rhead{2025-04-05}
\lfoot{}
\cfoot{DON'T PANIC}
\rfoot{\thepage/\pageref{LastPage}}
\renewcommand{\headrulewidth}{0.4pt}
\renewcommand{\footrulewidth}{0.4pt}

\usepackage{tcolorbox} % рамочки!

\usepackage{todonotes} % для вставки в документ заметок о том, что осталось сделать
% \todo{Здесь надо коэффициенты исправить}
% \missingfigure{Здесь будет Последний день Помпеи}
% \listoftodos - печатает все поставленные \todo'шки


% более красивые таблицы
\usepackage{booktabs}
% заповеди из докупентации:
% 1. Не используйте вертикальные линни
% 2. Не используйте двойные линии
% 3. Единицы измерения - в шапку таблицы
% 4. Не сокращайте .1 вместо 0.1
% 5. Повторяющееся значение повторяйте, а не говорите "то же"



\usepackage{fontspec}
\usepackage{polyglossia}

\setmainlanguage{russian}
\setotherlanguages{english}

% download "Linux Libertine" fonts:
% http://www.linuxlibertine.org/index.php?id=91&L=1
\setmainfont{Arial} % or Helvetica, Arial, Cambria
% why do we need \newfontfamily:
% http://tex.stackexchange.com/questions/91507/

\AddEnumerateCounter{\asbuk}{\russian@alph}{щ} % для списков с русскими буквами
\setlist[enumerate, 2]{label=\asbuk*),ref=\asbuk*}

%% эконометрические сокращения
\DeclareMathOperator{\Cov}{\mathbb{C}ov}
\DeclareMathOperator{\Corr}{\mathbb{C}orr}
\DeclareMathOperator{\Var}{\mathbb{V}ar}

\let\P\relax
\DeclareMathOperator{\P}{\mathbb{P}}

\DeclareMathOperator{\E}{\mathbb{E}}
% \DeclareMathOperator{\tr}{trace}
\DeclareMathOperator{\card}{card}
\DeclareMathOperator{\plim}{plim}
\DeclareMathOperator{\pCorr}{\mathrm{p}\mathbb{C}\mathrm{orr}}


\newcommand \hb{\hat{\beta}}
\newcommand \hs{\hat{\sigma}}
\newcommand \htheta{\hat{\theta}}
\newcommand \s{\sigma}
\newcommand \hy{\hat{y}}
\newcommand \hY{\hat{Y}}
\newcommand \e{\varepsilon}
\newcommand \he{\hat{\e}}
\newcommand \z{z}
\newcommand \hVar{\widehat{\Var}}
\newcommand \hCorr{\widehat{\Corr}}
\newcommand \hCov{\widehat{\Cov}}
\newcommand \cN{\mathcal{N}}
\newcommand \RR{\mathbb{R}}
\newcommand \NN{\mathbb{N}}
\newcommand{\cF}{\mathcal{F}}
\newcommand{\cH}{\mathcal{H}}


\begin{document}

% Кратко: очно. 

\begin{enumerate}
	
\item (15 баллов) При каких p и q указанный процесс будет являться белым шумом?

\[
X_t = A_t \cdot (-1)^{S_t},
\]

где:
\begin{itemize}
	\item $A_t$ — i.i.d. Бернулли ($P(A_t = 1) = p$),
	\item $S_t = \sum_{k=1}^{t-1} B_k$, где $B_k$ — i.i.d. Бернулли ($P(B_k = 1) = 	q$).
	\item $B_k$ и $A_t$ независимы
	\end{itemize}

\item (20 баллов) Пересвет Матрёшкин считает, что его данные могут быть описаны следующим уравнением:

%(1-0.2L)(1-0.3L) y_t = \epsilon_t (1-0.3L) (1-0.4L)(1-0.2L)
\[
y_t - 0.5 y_{t-1} + 0.06 y_{t-2} = 10 + \varepsilon_t - 0.9 \varepsilon_{t-1} + 0.08 \varepsilon_{t-2} - 0.024  \varepsilon_{t-3} 
\]

\begin{enumerate}
\item (4 балла) Проверьте, сколько у этого уравнения нестационарных, стационарных и стабильных решений? В ответе укажите три числа. Ответ обоснуйте.

\item (4 балла) Классифицируйте это уравнение как ARMA. Укажите p и q.

\item (4 балла) Найдите ACF(k) и PACF(k) в явном виде как функцию от k.

\item (4 балла) Найдите предел  $E(y_{T+h}| \mathcal{F}_T)$ при $h \to \infty$

\item (4 балла) Выразите $\varepsilon_t$ через предыдущие лаги $y_t$ (это называется $AR(\infty)$), или докажите, что это невозможно.

\end{enumerate}


\item (10 баллов) Рассмотрим модель $ETS(AAdN)$:
\[
\begin{cases}
	u_t  \sim \mathcal{N}(0;20) \\
	b_t = 0.8 b_{t-1} + 0.3 u_t \\
	\ell_t = \ell_{t-1} + 0.8 b_{t-1} + 0.2 u_t \\
	y_t = \ell_{t-1} + 0.8 b_{t-1} + u_t \\
\end{cases}
\]
где $\ell_{100} = 20$ and $b_{100} = 1$.
\begin{enumerate}
	\item (5 баллов) Найдите 95\% доверительный интервал $y_{102}$.
	\item (5 баллов) Приблизительно вычислите прогноз для $y_{2025}$.
\end{enumerate}


\item (15 баллов) Рассмотрим следующую VAR(2) - модель.

\[
\begin{pmatrix}
	y_{1,t} \\
	y_{2,t}
\end{pmatrix}
=
\begin{pmatrix}
	c_1 \\
	c_2
\end{pmatrix}
+
\begin{pmatrix}
	\phi_{11}^{(1)} & \phi_{12}^{(1)} \\
	\phi_{21}^{(1)} & \phi_{22}^{(1)}
\end{pmatrix}
\begin{pmatrix}
	y_{1,t-1} \\
	y_{2,t-1}
\end{pmatrix}
+
\begin{pmatrix}
	\phi_{11}^{(2)} & \phi_{12}^{(2)} \\
	\phi_{21}^{(2)} & \phi_{22}^{(2)}
\end{pmatrix}
\begin{pmatrix}
	y_{1,t-2} \\
	y_{2,t-2}
\end{pmatrix}
+
\begin{pmatrix}
	\varepsilon_{1,t} \\
	\varepsilon_{2,t}
\end{pmatrix}
\]

\begin{enumerate}
	\item (5 баллов) Кратко опишите, в чём отличия обычных IRF и ортогональных IRF?
	\item (5 баллов) В чём отличия интерпретации обычных IRF и кумулятивных?
	\item (5 баллов) Одним из преимуществ VAR(2) является тот факт, что через её оценку можно проводить тест Гранжера.  Гипотеза: $y_2$ является каузальным для $y_1$. Сформулируйте тест Гранжера относительно коэффициентов $\phi_{ij}^{(p)}$, выпишите нулевую и альтернативную гипотезы.
\end{enumerate}

\newpage

\item (15 баллов) У Лукоморья дуб зелёный. Златая цепь на дубе том. Пусть H -- размах кроны дуба в кошачьих шагах. Каждые H шагов кот учёный меняет направление своего движения и тема его повествования меняется. $\rho$ -- параметр креативности кота. $|\rho| < 1$, $\varepsilon_t $ -- белый шум. Тогда историю можно описать следующим процессом:

\[
x_t =
\begin{cases}
	 \rho x_{t-1} + \varepsilon_t, &mod( t / H) \neq 0\\
	\varepsilon_t,  &mod( t / H)  = 0,\\
\end{cases}
\]

где $mod(\cdot)$ означает остаток от деления. При каком условии на H процесс будет стационарным? 




\item (20 баллов) Рассмотрим модель парной регрессии с AR(1) ошибками:
\begin{equation*}
	y_t = \beta_0 + \beta_1 x_t + \varepsilon_t, \quad \varepsilon_t = \rho \varepsilon_{t-1} + u_t, \quad u_t \sim N(0, \sigma^2_u),
\end{equation*}
где $|\rho| < 1$, а $x_t$ — экзогенная переменная.

\begin{enumerate}
	\item (5 баллов) Найдите распределение $\varepsilon_1$
	\item (5 баллов) Найдите распределение $p(y_1 | x_1)$
	\item (5 баллов) Найдите распределение $p(y_t | y_{t-1}, x_t, x_{t-1	})$
	\item (5 баллов) Запишите полное правдоподобие $p(y_1, \ldots, y_T \mid x_1, \ldots, x_T) $
\end{enumerate}

\end{enumerate}


\end{document}

