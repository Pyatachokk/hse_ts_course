\documentclass[12pt,fleqn]{article}
\usepackage{../../vkCourseML}
%\usepackage{vkCourseML}
\hypersetup{unicode=true}
%\usepackage[a4paper]{geometry}
\usepackage[hyphenbreaks]{breakurl}

\interfootnotelinepenalty=10000
\newcommand{\dx}[1]{\,\mathrm{d}#1} % маленький отступ и прямая d


\begin{document}
	
	Наиболее широко используется аддитивная декомпозиция ряда. Во многом это связано с её простотой и наличием устойчивых алгоритмов декомпозиции. 
	$$y_t = t_t + s_t + c_t + e_t,$$
	
	где $t_t$ -- тренд, $s_t$ -- сезонность,  $c_t$ -- цикличность, а $e_t$ -- остаток.
	
	Аддитивность разложения вовсе необязательна, каждая из компонент технически может быть мультипликативной. Например, из модели ETS мы сможем достать мультипликативные компоненты, но всё же по большей части используют аддитивные подходы из-за простоты интерпретации.
	
	Обсудим смысл каждой из компонент.
	Дать им строгие формальные определения довольно затруднительно, поэтому они будут скорее интуитивными. \emph{Трендом} мы будем называть долгосрочное изменение уровня ряда.
	Можно условно подразделить тренды на восходящие, нисходящие и изменяющие своё направление. Нас же будет больше интересовать природа ряда. В разделе про нестационарные модели мы подробно обсудим, что тренды могут быть порождены как детерминированными функциями, так и стохастическими. \emph{Сезонность} это периодические колебания с фиксированным периодом. Например, продажи мороженого будут стабильно расти летом и падать зимой, а пассажиропоток в метро имеет довольно конкретные часы-пик, почти не изменяющиеся день ото дня. Но здесь важно не угождать нашему антропоцентризму и помнить, что, например, в астрономических задачах периоды сезонности могут не совпадать с земными. \emph{Цикличность} отличается от сезонности только нестабильным периодом и обычно большей длительностью колебаний. Хорошим примером могут быть циклы в любой крупной экономике, где полный период может занимать десятилетия.
	
	Декомпозиция может помочь для разных задач. С помощью неё удобно смотреть на временной ряд в разрезе и проводить эксплоративный анализ данных. Также в некоторых задачах требуется очистить ряд от той или иной компоненты. Например, макроэкономические ряды часто очищают от сезонности для анализа и прогнозирования. Наконец, прогнозировать ряд по частям может быть более удобно и надёжно. Так как все компоненты кроме остатка по построению довольно простые, их можно прогнозировать тривиальными моделями. Если тренд устойчивый, то его несложно экстраполировать линейной или экспоненциальной функцией, а с сезонностью неплохо справляются наивные модели. Цикличность тоже можно экстраполировать моделью сглаживания. Самое сложное обычно кроется в остатках, так как в этой компоненте будут зашиты нетривиальные зависимости. Именно на ряд остатков придётся строить сложную модель с дополнительными признаками и продвинутыми методами. Далее прогнозы всех компонент суммируются (или комбинируются по-другому если разложение не было аддитивным), и сумма будет прогнозом исходного ряда. Такие модели называют \emph{sandwich}.

	\section{Алгоритмы сглаживания}
	
	\subsection{Moving average}
	
	\subsection{OLS}
	
	\subsection{LOESS}
	
	\section{STL}
	
	\section{MSTL}

\end{document}