% arara: xelatex
\documentclass[12pt]{article}

% \usepackage{physics}

\usepackage{hyperref}
\hypersetup{
    colorlinks=true,
    linkcolor=blue,
    filecolor=magenta,      
    urlcolor=cyan,
    pdftitle={Overleaf Example},
    pdfpagemode=FullScreen,
    }

\usepackage{tikzducks}

\usepackage{tikz} % картинки в tikz
\usetikzlibrary{shapes, arrows, positioning}
\usepackage{microtype} % свешивание пунктуации

\usepackage{array} % для столбцов фиксированной ширины

\usepackage{indentfirst} % отступ в первом параграфе

\usepackage{sectsty} % для центрирования названий частей
\allsectionsfont{\centering}

\usepackage{amsmath, amsfonts, amssymb} % куча стандартных математических плюшек

\usepackage{comment}

\usepackage[top=2cm, left=1.2cm, right=1.2cm, bottom=2cm]{geometry} % размер текста на странице

\usepackage{lastpage} % чтобы узнать номер последней страницы

\usepackage{enumitem} % дополнительные плюшки для списков
%  например \begin{enumerate}[resume] позволяет продолжить нумерацию в новом списке
\usepackage{caption}

\usepackage{url} % to use \url{link to web}


\newcommand{\smallduck}{\begin{tikzpicture}[scale=0.3]
    \duck[
        cape=black,
        hat=black,
        mask=black
    ]
    \end{tikzpicture}}

\usepackage{fancyhdr} % весёлые колонтитулы
\pagestyle{fancy}
\lhead{Домашний праздник}
\chead{}
\rhead{}
\lfoot{}
\cfoot{}
\rfoot{}

\renewcommand{\headrulewidth}{0.4pt}
\renewcommand{\footrulewidth}{0.4pt}

\usepackage{tcolorbox} % рамочки!

\usepackage{todonotes} % для вставки в документ заметок о том, что осталось сделать
% \todo{Здесь надо коэффициенты исправить}
% \missingfigure{Здесь будет Последний день Помпеи}
% \listoftodos - печатает все поставленные \todo'шки


% более красивые таблицы
\usepackage{booktabs}
% заповеди из докупентации:
% 1. Не используйте вертикальные линни
% 2. Не используйте двойные линии
% 3. Единицы измерения - в шапку таблицы
% 4. Не сокращайте .1 вместо 0.1
% 5. Повторяющееся значение повторяйте, а не говорите "то же"


\setcounter{MaxMatrixCols}{20}
% by crazy default pmatrix supports only 10 cols :)


\usepackage{fontspec}
\usepackage{libertine}
\usepackage{polyglossia}

\setmainlanguage{russian}
\setotherlanguages{english}

% download "Linux Libertine" fonts:
% http://www.linuxlibertine.org/index.php?id=91&L=1
% \setmainfont{Linux Libertine O} % or Helvetica, Arial, Cambria
% why do we need \newfontfamily:
% http://tex.stackexchange.com/questions/91507/
% \newfontfamily{\cyrillicfonttt}{Linux Libertine O}

\AddEnumerateCounter{\asbuk}{\russian@alph}{щ} % для списков с русскими буквами
% \setlist[enumerate, 2]{label=\asbuk*),ref=\asbuk*}

%% эконометрические сокращения
\DeclareMathOperator{\Cov}{\mathbb{C}ov}
\DeclareMathOperator{\Corr}{\mathbb{C}orr}
\DeclareMathOperator{\Var}{\mathbb{V}ar}
\DeclareMathOperator{\col}{col}
\DeclareMathOperator{\row}{row}

\let\P\relax
\DeclareMathOperator{\P}{\mathbb{P}}

\DeclareMathOperator{\E}{\mathbb{E}}
% \DeclareMathOperator{\tr}{trace}
\DeclareMathOperator{\card}{card}

\DeclareMathOperator{\Convex}{Convex}
\DeclareMathOperator{\plim}{plim}

\newcommand{\cN}{\mathcal{N}}
\newcommand{\cF}{\mathcal{F}}

\newcommand{\RR}{\mathbb{R}}
\newcommand{\NN}{\mathbb{N}}
\newcommand{\hb}{\hat{\beta}}
\newcommand{\dPois}{\mathrm{Pois}}


\newcommand{\dExpo}{\mathrm{Expo}}




\begin{document}

\begin{enumerate}
    \item Все величины $(u_t)$, $v(0)$, $v(1)$, $v(2)$ независимы, одинаково распределены и равновероятно принимают значения $+1$ и $(-1)$.
    Рассмотрим процесс 
    \[
    \begin{cases} 
        r_t = t \text{ mod } 3, \text{ (остаток от деления }t \text{ на 3)} \\
        y_t = 100 v(r_t) + u_t + 0.5u_{t-1}.   
    \end{cases}
    \]
    \begin{enumerate}
        \item {[2]} Нарисуйте пару «типичных» траекторий процесса $(y_t)$. 
        \item {[3]} Является ли процесс $(y_t)$ слабо стационарным?
        \item {[3]} Представим ли данный процесс в виде $MA(\infty)$ процесса?
        \item {[2]} Правда ли, что выборочная ковариации сходится к теоретической,
        \[
        \plim \sum_{t=2}^T y_t y_{t-1} / T = \Cov(y_1, y_2)?
        \]
    \end{enumerate}

    \item Динамика количества ежей в лесу $(y_t)$ описывается полугодовым $ETS(AAdA)$ процессом:
    \[
    \begin{cases}
        u_t \sim \cN(0, 9) \\
        s_t = s_{t-2} + 0.1 u_t \\
        b_t = 0.9b_{t-1} + 0.1 u_t \\
        \ell_t = \ell_{t-1} + 0.9b_{t-1} + 0.2 u_t \\
        y_t = \ell_{t-1} + 0.9b_{t-1} + s_{t-2} + u_t \\
    \end{cases}    
    \]
    Известно, что $s_{100} = 2$, $s_{99} = -3$, $\ell_{100} = 200$, $b_{100} = 1$.
    \begin{enumerate}
      \item {[6]} Постройте 95\%-й предиктивный интервал количества ежей $y_{102}$ через год.
      \item {[4]} Запишите эту модель в виде $A(L) y_t = B(L) u_t$, где $A(L)$ и $B(L)$ взаимно-простые лаговые многочлены.
    \end{enumerate}
  
    \item Величины $W_1$, $W_2$ независимы и имеют функцию плотности $f(w) = 2w$ на отрезке $[0, 1]$.
    Определим $X_1 = \min \{W_1, W_2\}$ и $X_2 = \max \{W_1, W_2\}$.
    \begin{enumerate}
        \item {[3]} Найдите функцию распределения $F_1$ величины $X_1$ и функцию распределения $F_2$ величины $X_2$.
        \item {[4]} Найдите копулу $C(u_1, u_2)$ для пары $(X_1, X_2)$.
        \item {[3]} Найдите условную вероятность $\P(F_1(X_1) \leq u_1 \mid F_2(X_2) = u_2)$.
    \end{enumerate}


    \newpage

    \item Рассмотрим разностное уравнение $y_t = 10 + 0.5 y_{t-1} + u_{t} + 2 u_{t-1}$, где $(u_t)$ — белый шум. 
    \begin{enumerate}
        \item {[2]} Сколько нестационарных решений у этого уравнения? Привидете в качестве примера хотя бы одно нестационарное решение. 
    \end{enumerate}
    Винни-Пух использует в качестве модели для численности пчёл единственное стационарное решение этого уравнения. 
    \begin{enumerate}[resume]
        \item {[3]} Выпишите явно решение, которое использует Винни-Пух. 
        \item {[3]} Сможет ли Винни-Пух восстановить $u_0$, если он знает весь бесконечный ряд $y_0$, $y_{-1}$, $y_{-2}$, \dots?
        \item {[2]} Предложите уравнение, единственное стационарное решение которого имеет ожидание и автоковариационную функцию идентичные ожиданию и автоковариационной функции исходного процесса,
        но при этом по прошлым значениям нового процесса можно восстановить ненаблюдаемое значение случайного шока. 
    \end{enumerate}

    \item   Строго стационарный процесс $(u_t)$ описывается $ARCH(1)$ моделью $\sigma^2_t = 3 + 0.2 u_{t-1}^2$, где 
    $u_t = \sigma_t \nu_t$ и шумы $\nu_t \sim \cN(0, 1)$ независимы.
    \begin{enumerate}
        \item {[3]} Найдите $\E(u_t)$, $\Var(u_t)$.
        \item {[5]} Постройтие 95\%-й предиктивный интервал для $u_{101}$ если $u_{100} = -1$.
        \item {[2]} Верно ли, что условное распределение $u_{102}$ при $u_{100} = -1$ является нормальным?
    \end{enumerate}

    \item Рассмотрим двумерный слабо стационарный $VAR(2)$ процесс $y_t = (y_{1t}, y_{2t})$, являющийся решением уравнения
    \[
    y_t = \begin{pmatrix}
    4 \\ 
    11 \\    
    \end{pmatrix} + 
    \begin{pmatrix}
      0.2  & 0.1 \\
      0  & 0.2 \\
    \end{pmatrix} y_{t-1} + 
    \begin{pmatrix}
      0.2  & 0 \\
      0.1  & 0.2 \\
    \end{pmatrix} y_{t-2} + u_t,
    \]
    где двумерный белый шум $u_t = (u_{1t}, u_{2t})$ c $\E(u_t) = 0$ и $\Var(u_t) = \begin{pmatrix}
        4 & 0 \\
        0 & 9 \\
    \end{pmatrix}$.
    \begin{enumerate}
        \item {[2]} Найдите $\E(y_t)$.
        \item {[4]} Найдите первые два значения кросс-ковариационной функции $\gamma_{12}(k) = \Cov(y_{1,t}, y_{2, t-k})$: $\gamma_{12}(1)$ и $\gamma_{12}(2)$.
        \item {[4]} Перепишите данный процесс в виде $VAR(1)$ процесса более высокой размерности, $w_t = c + A w_{t-1} + v_t$.
        Явно укажите матрицу $A$, вектор $c$, выразите вектор $w_t$ через вектор $y_t$, выпишите ковариационную матрицу белого шума $\Var(v_t)$.
    \end{enumerate}

\end{enumerate}


\end{document}

