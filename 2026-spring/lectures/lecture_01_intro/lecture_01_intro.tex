%!TEX TS-program = xelatex

% Официальный шаблон презентации НИУ ВШЭ в beamer (LaTeX)
% Версия 2.0
% Язык — русский   
% Автор шаблона - Данил Фёдоровых (fedorovykh@gmail.com)

%%% Для корректной работы шаблона необходима 
%%% установка в систему бесплатного шрифта HSE Sans
%%% https://www.hse.ru/info/brandbook/#font


\documentclass[aspectratio=169]{beamer}

\newbool{russian}
\booltrue{russian}  % Загружает русскоязычный логотип ВШЭ
\usepackage{HSE-theme/beamerthemeHSE} % Подгружаем тему

%%% Работа с русским языком и шрифтами
\usepackage[english,russian]{babel}   % загружает пакет многоязыковой вёрстки
\usepackage[no-math]{fontspec}      % подготавливает загрузку шрифтов Open Type, True Type и др.
	\setsansfont{HSE Sans} 
	\setmonofont{Courier New}
\usepackage{mathspec}
	\setmathsfont(Digits,Latin,Greek)[Numbers={Lining,Proportional}]{HSE Sans}
	\setmathrm[Numbers={Lining,Proportional}]{HSE Sans}
\uselanguage{russian}
\languagepath{russian}
\deftranslation[to=russian]{Theorem}{Теорема}
\deftranslation[to=russian]{Definition}{Определение}
\deftranslation[to=russian]{Definitions}{Определения}
\deftranslation[to=russian]{Corollary}{Следствие}
\deftranslation[to=russian]{Fact}{Факт}
\deftranslation[to=russian]{Example}{Пример}
\deftranslation[to=russian]{Examples}{Примеры}

\usepackage{blindtext} 		% Случайный текст
\graphicspath{{images/}}  	% Папка с картинками

%%% Информация об авторе и выступлении
\title[Заголовок]{Лекция 1} 
\subtitle{Структура курса. \\Задачи моделирования временных рядов.}
\author[Матвей Зехов]{Матвей Зехов \\ \smallskip \scriptsize mzekhov@hse.ru}
\institute{Департамент больших данных и информационного поиска}
\date{\today}


\begin{document}	

\frame[plain]{\titlepage}	% Титульный слайд


\begin{frame}
\frametitle{Формула оценки}
	
	$$\text{Итог} = \text{Округление}(0.75 \cdot \text{Накоп.} + 0.25 \cdot \text{Экз.})$$
	
	где:
	$$\text{Накоп.} = \frac{2}{3} \cdot \text{ДЗ} + \frac{1}{3} \cdot \text{КР}$$
	
	$$\text{ДЗ} = 0.25 \cdot \text{ДЗ}_1 + 0.25 \cdot \text{ДЗ}_2 + 0.25 \cdot \text{ДЗ}_3 + 0.25 \cdot \text{ДЗ}_4$$
	
	\begin{itemize}
		\item ДЗ — средняя оценка за домашние задания (4 домашних задания)
		\item КР — оценка за контрольную работу
		\item Экз. — оценка за экзамен
		\item Округление — арифметическое, применяется только к итоговой оценке
		\item Одно из домашних заданий является теоретическим
	\end{itemize}
\end{frame}

\begin{frame}
\frametitle{Автоматы}
	Всем студентам может быть автоматом выставлена оценка за экзамен, равная 
	$$\text{Экз.}_{\text{автомат}} = \min(7, \text{Накоп.})$$
	
	Итоговая оценка будет рассчитана по стандартной формуле:
	$$\text{Итог} = \text{Округление}(0.75 \cdot \text{Накоп.} + 0.25 \cdot \text{Экз.}_{\text{автомат}})$$
	
	При явке на экзамен эта возможность аннулируется. 
	
\end{frame}

\begin{frame}
\frametitle{Дополнительные условия}
	\begin{itemize}
		\item При невозможности выполнения любого из ДЗ по уважительной причине и при наличии соответствующей справки, студент вправе перенести вес ДЗ на Экзамен. Для этого необходимо передать справку в учебную часть, а также уведомить семинариста. Автомат в таком случае не может быть выставлен.
		
		\item При пропуске КР по уважительной причине вес КР переносится на Экзамен. Для этого необходимо передать справку в учебную часть, а также уведомить семинариста. Автомат в таком случае не может быть выставлен.
	\end{itemize}
\end{frame}

\begin{frame}
\frametitle{Домашние задания: Общие правила}
	\begin{itemize}
		\item Домашние задания сдаются в Anytask
		\item Инвайт будет выслан в групповой чат
		\item Название основного файла должно быть в формате: \\
		\texttt{Surname\_name\_HW\#.ipynb} (или {Surname\_name\_HW\#.pdf} для теордз)  где \# - номер задания
		\item За несоответствие имени файла формату предусмотрен штраф 0.5 балла из 10.
	\end{itemize}
\end{frame}

\begin{frame}
\frametitle{Домашние задания: Дедлайны}
	\begin{itemize}
		\item Мягких дедлайнов по ДЗ нет. Все дедлайны жёсткие.
		\item Студент имеет право два раза за курс просрочить дедлайн по любому из ДЗ (практическому или теоретическому) на 24 часа без штрафа
		\item Или можно просрочить одно ДЗ на 48 часов
		\item Для этого необходимо оставить соответствующий комментарий в anytask
	\end{itemize}
\end{frame}

\begin{frame}
\frametitle{Домашние задания: Плагиат}
	\begin{itemize}
		\item При обнаружении плагиата оценки за домашнее задание обнуляются всем задействованным в списывании студентам
		\item Подается докладная записка в деканат
		\item При повторном списывании деканат имеет право отчислить студента
	\end{itemize}
\end{frame}

\begin{frame}
\frametitle{Домашние задания: Использование LLM}
	\begin{itemize}
		\item Использование LLM в качестве помощника для домашнего задания не запрещается для небольших элементов кода
		\item Каждый сгенерированный автоматически элемент домашнего задания должен быть помечен явно
		\item Запрещается применение LLM для решения теоретического ДЗ и любых элементов практических ДЗ, требующих теоретических выводов
		\item Любые неразмеченные элементы, выполненные LLM, оцениваться не будут
		\item При попытке несамостоятельного выполнения существенной части ДЗ (на усмотрение лектора или семинариста) работа обнуляется, а также подается докладная записка в деканат
	\end{itemize}
\end{frame}

\begin{frame}
\frametitle{Контрольная работа}
	\begin{itemize}
		\item Письменная
		\item Без чит-листов
		\item Очно
		\item Теоретические задачи
		\item Ориентировочно -- первая неделя после весенней сессии
	\end{itemize}
\end{frame}

\begin{frame}
\frametitle{Экзамен}
	\begin{itemize}
		\item Устный
		\item Без чит-листов
		\item Очно
		\item Теоретические вопросы и небольшие задачи
		\item Список теоретических вопросов появится не позднее чем за две недели до экзамена
	\end{itemize}
\end{frame}

\begin{frame}
\frametitle{Доклады по статьям}
	На предпоследней неделе курса лекция и семинар будут посвящены обзорам современных статей по прогнозированию и моделированию.
	
	\vspace{0.3cm}
	\textbf{Формат:}
	\begin{itemize}
		\item 7--8 докладов по 20 минут
		\item Один студент -- одна статья
	\end{itemize}
	
	\vspace{0.1cm}
	\textbf{Оценка:}
	\begin{itemize}
		\item Не входит в формулу оценки
		\item Дополнительные баллы к ДЗ или КР.
		\item 3 балла к любому ДЗ/1 балл к КР и 1 к ДЗ/1.5 балла к КР
	\end{itemize}
	
	\vspace{0.1cm}
	\textbf{Организация:}
	\begin{itemize}
		\item Список статей будет опубликован после 3-го ДЗ
		\item По желанию можно предложить статью не из списка (по согласованию с лектором)
		\item Лектор и семинарист участвуют в дискуссии и модерируют
	\end{itemize}
\end{frame}

\begin{frame}
\frametitle{Дополнительные баллы}
	
	\begin{itemize}
		\item Teхать варианты предыдущих лет
		\item Teхать решения задач в задачнике
		\item Искать ошибки в уже существующих решениях
	\end{itemize}
	
\end{frame}

\begin{frame}
\frametitle{Основные определения}
	\begin{definition}[Случайный процесс]
	Случайный процесс --- это семейство случайных величин $\{X_t, t \in T\}$, индексированных параметром $t$, который обычно интерпретируется как время. Здесь $T$ --- множество индексов (чаще всего $T \subseteq \mathbb{R}$ или $T \subseteq \mathbb{Z}$).
	\end{definition}
	
	\vspace{0.3cm}
	
	\begin{definition}[Временной ряд]
	Временной ряд --- это последовательность наблюдений $\{x_1, x_2, \dots, x_n\}$, полученных в последовательные моменты времени, где каждое наблюдение $x_t$ является реализацией случайной величины $X_t$ из некоторого случайного процесса.
	\end{definition}
	
	\vspace{0.3cm}
	
	Таким образом, временной ряд --- это выборка (реализация) случайного процесса.
\end{frame}

\begin{frame}
\frametitle{Прогнозирование}
\framesubtitle{Один ряд}
	
	\begin{definition}[Прогнозирование]
	Прогнозирование --- это процесс предсказания будущих значений временного ряда на основе наблюдаемых прошлых и настоящих значений.
	\end{definition}
	
	\vspace{1cm}
	
	\begin{center}
	\includegraphics[width=0.7\textwidth]{lecture_01_intro_01.png}
	\end{center}
	
\end{frame}

\begin{frame}
\frametitle{Принципы прогнозирования}
	
	\textbf{Использование исторических данных:}
	\begin{itemize}
		\item Прогнозирование основано на анализе прошлых значений временного ряда
		\item Не предполагает знания будущих событий (no future peeking)
		\item Используются только доступные на момент прогноза данные
	\end{itemize}
	
	\vspace{0.3cm}
	
	\textbf{Сохранение паттернов:}
	\begin{itemize}
		\item Предполагается, что некоторые закономерности сохраняются во времени
		\item Используются повторяющиеся паттерны (сезонность, тренды)
		\item Учитываются циклические и структурные зависимости
	\end{itemize}
	
	\vspace{0.3cm}
	
	\textbf{Выбор горизонта прогнозирования:}
	
	С увеличением горизонта прогнозирования неопределенность прогноза возрастает, что приводит к снижению его точности. Краткосрочные прогнозы обычно более точны, чем долгосрочные.
	
\end{frame}

\begin{frame}
\frametitle{Типы прогнозирования}
	
	Прогнозирование можно классифицировать по различным критериям:
	\begin{itemize}
		\item По типу целевой переменной
		\item По горизонту прогнозирования
		\item По количеству прогнозируемых рядов
		\item По частоте обновления модели
	\end{itemize}
	
\end{frame}

\begin{frame}
\frametitle{Типы прогнозирования по таргету}
	
	По типу целевой переменной:
	\begin{itemize}
		\item Точечное прогнозирование (Point Forecasting)
		\item Прогнозирование квантилей (Quantile Forecasting)
		\item Интервальное прогнозирование (Interval Forecasting)
		\item Прогнозирование волатильности
		\item Прогнозирование плотности распределения
		\item Прогнозирование аномалий
		\item Сценарное прогнозирование
	\end{itemize}
	
\end{frame}

\begin{frame}
\frametitle{Прогнозирование по горизонту}
	
	\begin{columns}
		\column{0.5\textwidth}
		\textbf{Одношаговое прогнозирование (Single-step)}
		\begin{itemize}
			\item Прогнозируется только следующее значение
			\item Используются доступные данные до текущего момента
			\item Высокая точность
			\item Простая модель
		\end{itemize}
		
		\column{0.5\textwidth}
		\textbf{Многошаговое прогнозирование (Multi-step)}
		\begin{itemize}
			\item Прогнозируется несколько будущих значений
			\item Может использовать предыдущие прогнозы
			\item Накопление ошибки
			\item Более сложная модель
		\end{itemize}
	\end{columns}
	
\end{frame}


\begin{frame}
\frametitle{Заполнение пропусков}
	
	\begin{itemize}
		\item \textbf{Простые методы} — заполнение средним, медианой, переносом вперёд/назад (LOCF/BOCF).
		\item \textbf{Интерполяция} — линейная, полиномиальная, сплайновая, временная.
		\item \textbf{Статистические модели} — ARIMA, SARIMA, экспоненциальное сглаживание, модели пространства состояний.
		\item \textbf{Машинное обучение} — использование алгоритмов ML (Random Forest, XGBoost и др.) на основе лагов и признаков.
		\item \textbf{Иные методы} — глубокое обучение (LSTM, Transformer), вероятностные методы (Gaussian Processes), гибридные подходы, множественная импутация, методы на основе DTW, пространственно-временные модели, методы для категориальных и нерегулярных рядов.
	\end{itemize}
	
\end{frame}

\begin{frame}
\frametitle{Частота и периодичность временных рядов}
	
	\begin{definition}[Частота временного ряда]
	Частота временного ряда — это количество наблюдений в единицу времени (например, ежедневные, еженедельные, ежемесячные данные).
	\end{definition}
	
	\vspace{0.5cm}
	
	\begin{definition}[Периодичность временного ряда]
	Периодичность временного ряда — это регулярность появления наблюдений:
	\begin{itemize}
		\item \textbf{Периодические ряды} — данные поступают через равные промежутки времени (например, ежедневные продажи).
		\item \textbf{Спорадические ряды} — данные поступают нерегулярно (например, транзакции в интернет-магазине).
	\end{itemize}
	\end{definition}
	
\end{frame}

\begin{frame}
\frametitle{Периоды сезонности}
	
	\begin{definition}[Период сезонности]
	Период сезонности — это длина временного интервала, после которого повторяется определённый паттерн в данных.
	\end{definition}
	
	\vspace{0.5cm}
	
	Примеры периодов сезонности:
	\begin{itemize}
		\item \textbf{Ежедневные данные}: период сезонности может составлять 7 дней (недельная сезонность) или 365 дней (годовая сезонность).
		\item \textbf{Ежемесячные данные}: период сезонности может составлять 12 месяцев (годовая сезонность).
		\item \textbf{Ежечасные данные}: период сезонности может составлять 24 часа (дневная сезонность) или 168 часов (недельная сезонность).
	\end{itemize}
	
\end{frame}

\begin{frame}
\frametitle{Агрегация временных рядов}
	
	\begin{definition}[Агрегация временного ряда]
	Агрегация временного ряда — это процесс объединения данных временного ряда на более высоком уровне детализации (например, суммирование или усреднение данных по дням для получения месячных данных).
	\end{definition}
	
	\vspace{0.5cm}
	
	\textbf{Примеры агрегации:}
	\begin{itemize}
		\item Агрегация ежедневных продаж в месячные для стратегического планирования
		\item Суммирование часовых данных потребления электроэнергии для получения суточных показателей
		\item Усреднение минутных данных цен на акции для получения часовых графиков
	\end{itemize}
	
	
\end{frame}

\begin{frame}
\frametitle{Дезагрегация временных рядов}
	
	\begin{definition}[Дезагрегация временного ряда]
	Дезагрегация временного ряда — это процесс разбиения данных временного ряда с более высокого уровня агрегации на более низкий уровень детализации (например, разбиение месячных данных на дневные).
	\end{definition}
	
	\vspace{0.5cm}
	
	\textbf{Примеры дезагрегации:}
	\begin{itemize}
		\item Дезагрегация годового бюджета в месячные планы для операционного управления
		\item Распределение квартальных прогнозов продаж по месяцам
		\item Разбиение недельных данных трафика на дневные показатели
	\end{itemize}
	
	
\end{frame}

\begin{frame}
\frametitle{Детекция выбросов в временных рядах}
	
	\begin{definition}[Выброс]
	Выброс — это отдельное наблюдение в временной последовательности, которое значительно отличается от соседних значений и не соответствует общему паттерну ряда.
	\end{definition}
	
	\vspace{0.5cm}
	
	Причины возникновения выбросов:
	\begin{itemize}
		\item Ошибки измерения или сбои в системе сбора данных
		\item Редкие события или аномальные ситуации
		\item Внешние воздействия или шоки
	\end{itemize}
	
	\vspace{0.5cm}
	
	Методы детекции выбросов:
	\begin{itemize}
		\item Статистические методы: Z-оценка, межквартильный размах (IQR)
		\item Методы на основе скользящего окна: сравнение со средним значением в окне
		\item Методы машинного обучения: изолирующий лес (Isolation Forest)
	\end{itemize}
	
\end{frame}

\begin{frame}
\frametitle{Фильтрация временных рядов}
	
	Фильтрация временных рядов — это процесс удаления шума из данных для выделения полезного сигнала и улучшения качества анализа.
	
	\vspace{0.5cm}
	
	Основные подходы к фильтрации:
	\begin{itemize}
		\item \textbf{Скользящее среднее} — усреднение значений в скользящем окне
		\item \textbf{Медианный фильтр} — замена значений медианой в окне, эффективен для удаления выбросов
		\item \textbf{Фильтр Калмана} — рекурсивный алгоритм оптимальной фильтрации, использующий модель системы и статистические характеристики шума
	\end{itemize}

\end{frame}


\begin{frame}
\frametitle{Декомпозиция временного ряда: Trend (Тренд)}
	
	\begin{itemize}
		\item Медленно изменяющийся уровень ряда, обычно выделяемый через сглаживание
		\item Может быть линейным или нелинейным
	\end{itemize}
	
	\vspace{1cm}
	
	\begin{center}
	\includegraphics[width=0.6\textwidth]{lecture_01_intro_03.png}
	\end{center}
	
\end{frame}

\begin{frame}
\frametitle{Декомпозиция временного ряда: Seasonality (Сезонность)}
	
	\begin{itemize}
		\item Повторяющиеся колебания значений временного ряда
		\item Возникают через фиксированные промежутки времени
	\end{itemize}
	
	\vspace{1cm}
	
	\begin{center}
	\includegraphics[width=0.6\textwidth]{lecture_01_intro_04.png}
	\end{center}
	
\end{frame}

\begin{frame}
\frametitle{Декомпозиция временного ряда: Cyclicity (Цикличность)}
	
	\begin{itemize}
		\item Колебания значений временного ряда, не имеющие фиксированного периода
		\item Возникают через нерегулярные промежутки времени
		\item Часто связаны с экономическими или бизнес-циклами
	\end{itemize}
	
	\vspace{1cm}
	
	\begin{center}
	\includegraphics[width=0.6\textwidth]{lecture_01_intro_05.png}
	\end{center}
	
\end{frame}

\begin{frame}
\frametitle{Декомпозиция временного ряда}
	
	\begin{center}
	\includegraphics[width=0.7\textwidth]{lecture_01_intro_02.png}
	\end{center}
	
\end{frame}

\begin{frame}
\frametitle{Связи между задачами временных рядов}
	
	Различные задачи временных рядов тесно связаны между собой:
	
	\vspace{0.5cm}
	
	\textbf{Детекция аномалий и прогнозирование:}
	\begin{itemize}
		\item Обнаружение выбросов улучшает качество прогнозов
		\item Аномальные значения могут искажать модель прогнозирования
		\item Предварительная очистка данных повышает точность
	\end{itemize}
	
	\vspace{0.3cm}
	
	\textbf{Фильтрация и все последующие задачи:}
	\begin{itemize}
		\item Удаление шума улучшает результаты всех методов анализа
		\item Фильтрация является важным этапом предобработки
		\item Повышает стабильность моделей
	\end{itemize}
	
	\vspace{0.3cm}
	
	\textbf{Декомпозиция и выбор моделей:}
	\begin{itemize}
		\item Понимание компонентов ряда помогает выбрать подходящие модели
		\item Разные компоненты могут требовать разных подходов
		\item Упрощает интерпретацию результатов
	\end{itemize}
	
\end{frame}

\begin{frame}
\frametitle{Классификация временных рядов}
	
	\begin{definition}[Классификация временных рядов]
	Классификация временных рядов — это задача машинного обучения, в которой временной ряд относится к одной из заранее определённых категорий или классов.
	\end{definition}
	
	\vspace{0.5cm}
	
	Примеры применения:
	\begin{itemize}
		\item Классификация типов ЭКГ сигналов
		\item Определение типа потребительского поведения по временным рядам покупок
		\item Классификация жестов по данным с акселерометра
	\end{itemize}
	
	Основные подходы:
	\begin{itemize}
		\item Извлечение признаков с последующим применением классических алгоритмов (SVM, Random Forest)
		\item Использование глубокого обучения (CNN, RNN)
		\item Методы на основе расстояний между рядами (DTW)
	\end{itemize}
	
\end{frame}

\begin{frame}
\frametitle{Кластеризация временных рядов}
	
	\begin{definition}[Кластеризация временных рядов]
	Кластеризация временных рядов — это задача unsupervised learning, в которой временные ряды группируются по схожести их паттернов без заранее определённых меток классов.
	\end{definition}
	
	\vspace{0.5cm}
	
	Примеры применения:
	\begin{itemize}
		\item Группировка клиентов по схожести временных рядов покупок
		\item Выявление типичных паттернов потребления электроэнергии
		\item Сегментация временных рядов цен на акции
	\end{itemize}
	
	\vspace{0.5cm}
	
	Основные подходы:
	\begin{itemize}
		\item Кластеризация на основе признаков (k-means, DBSCAN)
		\item Кластеризация на основе расстояний между рядами (k-means с DTW)
		\item Использование автоэнкодеров для снижения размерности с последующей кластеризацией
	\end{itemize}
	
\end{frame}

\begin{frame}
\frametitle{Поиск похожих временных рядов}
	
	\begin{definition}[Поиск похожих временных рядов]
	Поиск похожих временных рядов — это задача нахождения временных рядов из базы данных, которые имеют схожие паттерны с заданным временным рядом.
	\end{definition}
	
	\vspace{0.5cm}
	
	Примеры применения:
	\begin{itemize}
		\item Поиск похожих паттернов на финансовых рынках
		\item Рекомендация товаров на основе схожести временных рядов покупок
		\item Поиск аномалий через сравнение с нормальными паттернами
	\end{itemize}
	
	\vspace{0.5cm}
	
	Основные подходы:
	\begin{itemize}
		\item Использование расстояний между рядами (Euclidean, DTW, LCSS)
		\item Поиск на основе признаков (хэширование, индексирование)
		\item Использование представлений временных рядов (embeddings) для поиска в векторном пространстве
	\end{itemize}
	
\end{frame}

\begin{frame}
\frametitle{Особенности временных рядов}
	
	\begin{columns}
		\column{0.33\textwidth}
		\textbf{Ограниченность данных:}
		\begin{itemize}
			\item Временных рядов в целом не очень много
			\item Собираются хуже, чем текстовые данные
			\item Часто отсутствуют большие датасеты
		\end{itemize}
		
		\column{0.33\textwidth}
		\textbf{Разнородность структуры:}
		\begin{itemize}
			\item Сильно различаются по структуре
			\item Разные частоты, длины, сезонности
			\item Требуют индивидуальной настройки
		\end{itemize}
		
		\column{0.33\textwidth}
		\textbf{Краткость рядов:}
		\begin{itemize}
			\item Часто короткие, особенно для низких частот
			\item Недостаток исторических данных
			\item Требуются специальные методы для малых выборок
		\end{itemize}
	\end{columns}
	
\end{frame}

\begin{frame}
\frametitle{Классические статистические модели (часть 1)}
	
	Классические статистические модели включают в себя:
	\begin{itemize}
		\item ETS (Exponential Smoothing State Space Model)
		\item SARIMAX (Seasonal AutoRegressive Integrated Moving Average with eXogenous regressors)
		\item GARCH (Generalized AutoRegressive Conditional Heteroskedasticity)
		\item Иные линейные и нелинейные модели (TAR, ARFIMA, ...)
	\end{itemize}
	
	\vspace{0.5cm}
	
	\textbf{Когда применять:}
	\begin{itemize}
		\item Мало данных для обучения
		\item Требуется высокая скорость работы модели
		\item Необходимо получить бенчмарк для сравнения с более сложными моделями
		\item Интерпретируемость модели важнее точности
	\end{itemize}
	
\end{frame}

\begin{frame}
\frametitle{Классические статистические модели (часть 2)}
	
	Преимущества:
	\begin{itemize}
		\item Хорошо изучены и теоретически обоснованы
		\item Быстрая обучаемость и предсказание
		\item Интерпретируемость параметров
	\end{itemize}
	
	\vspace{0.5cm}
	
	Недостатки:
	\begin{itemize}
		\item Ограниченная гибкость в моделировании сложных зависимостей
		\item Требуют определенных предположений о структуре данных
		\item Для некоторых моделей может потребоваться ручная предобработка
	\end{itemize}
	
\end{frame}

\begin{frame}
\frametitle{Классические методы машинного обучения (часть 1)}
	
	Классические методы машинного обучения на табличных данных:
	\begin{itemize}
		\item Линейная регрессия (Linear Regression)
		\item Градиентный бустинг (XGBoost, LightGBM, CatBoost)
		\item Случайный лес (Random Forest)
		\item Метод опорных векторов (SVM) и другие
	\end{itemize}
	
	\vspace{0.5cm}
	
	\textbf{Когда применять:}
	\begin{itemize}
		\item Среднее или большое количество данных
		\item Требуется гибкость в моделировании сложных зависимостей
		\item Необходимо учитывать внешние факторы (экзогенные переменные)
		\item Есть достаточно времени и ресурсов на ресёрч
	\end{itemize}
	
\end{frame}

\begin{frame}
\frametitle{Классические методы машинного обучения (часть 2)}
	
	Преимущества:
	\begin{itemize}
		\item Высокая гибкость в моделировании зависимостей
		\item Хорошо работают с признаками, извлеченными из временных рядов
		\item Относительно простая интерпретация
	\end{itemize}
	
	\vspace{0.5cm}
	
	Недостатки:
	\begin{itemize}
		\item Требуют ручного извлечения признаков
		\item Могут иметь проблемы с долгосрочным прогнозированием
		\item Не всегда эффективны при сложных временных зависимостях
	\end{itemize}
	
\end{frame}

\begin{frame}
\frametitle{Модели глубокого обучения (часть 1)}
	
	Модели глубокого обучения включают в себя:
	\begin{itemize}
		\item Рекуррентные нейронные сети (RNN)
		\item Долгая краткосрочная память (LSTM)
		\item Gated Recurrent Units (GRU)
		\item Трансформеры (Transformers)
		\item Сверточные нейронные сети (CNN) для временных рядов
	\end{itemize}
	
	\vspace{0.5cm}
	
	\textbf{Когда применять:}
	\begin{itemize}
		\item Большое количество данных
		\item Высокая частота временных рядов
		\item Сложные нелинейные зависимости и паттерны
		\item Однородные наборы данных
	\end{itemize}
	
\end{frame}

\begin{frame}
\frametitle{Модели глубокого обучения (часть 2)}
	
	Преимущества:
	\begin{itemize}
		\item Высокая точность на больших наборах данных
		\item Автоматическое извлечение признаков
		\item Хорошо работают с многомерными временными рядами
	\end{itemize}
	
	\vspace{0.5cm}
	
	Недостатки:
	\begin{itemize}
		\item Требуют много данных для обучения
		\item Сложность в очистке и подготовке данных
		\item Часто нестабильны и требуют тонкой настройки
		\item Низкая интерпретируемость модели
	\end{itemize}
	
\end{frame}

\end{document}
